\section{Przebieg pracy}

W~tym podrozdziale opiszę fazy tworzenia projektu i~jego rozwoju.

\subsection{Przygotowanie środowiska pracy}

Pierwszym krokiem do~stworzenia systemu aukcyjnego w~technologii Ruby~on~Rails jest przygotowanie środowiska, w~którym projekt bedzie tworzony. Pominięto tutaj proces instalacji systemu operacyjnego.

\subsubsection{Instalacja narzędzi}

Większość narzędzi wymienionych w~rozdziale \ref{narzedzia} Narzędzia można zainstalować wpisując w~konsoli następujące polecenie:


\texttt{sudo apt-get install zsh screen ack-grep vim ruby git git-core tig ticgit ticgitweb sqlite3 sqlite3-dev sqliteman curl}


(Polecenie to~zadziała na~Ubuntu~11.10 oraz na~Debianie~Squeeze. Na~innych dystrybucjach zamiast \texttt{apt-get} należy użyć innego manadżera pakietów dostępnego dla danej dystrybucji. Oczywiście nazwy pakietów także mogą się różnić)

\subsubsection{Zarządzanie wersją Ruby -- RVM}

W~celu zainstalowania \textit{RVM} należy wpisać w~konsoli:


\texttt{bash -s stable < <(curl -s https://raw.github.com/wayneeseguin/rvm/master/binscripts/rvm-installer)}


Aby zainstalować Ruby w~wersji 1.9.3 należy wpisać:


\texttt{rvm install 1.9.3}


Po~konfiguracji i~kompilacji zadanej wersji Ruby należy tylko przełączyć się na~nią:


\texttt{rvm --default use 1.9.3}


Polecenie \texttt{ruby -v} powinno wypisać \texttt{ruby 1.ruby9.ruby93p0 (2011-10-30 revision 33570) [x86\_64-linux]}.

\subsubsection{Instalacja Ruby~on~Rails oraz niezbędnych wtyczek}

Ruby~on~Rails instalujemy wpisując:


\texttt{gem install rails}


W~ten sposób można już utworzyć pierwszy, podstawowy projekt Ruby~on~Rails:


\texttt{rails new auctioneer}


Polecenie to~utworzy katalog auctioneer zawierający podstawowe elementy każdego projektu Ruby~on~Rails.

\lstset{language=bash, caption=Zawartość katalogu \textit{auctioneer}, basicstyle=\ttfamily\footnotesize, captionpos=b, backgroundcolor=\color{LightGray}} \label{code.railsdir}
\begin{lstlisting}
app
app/models
app/views
app/views/layouts
app/controllers
app/helpers
app/assets
app/assets/stylesheets
app/assets/javascripts
app/assets/images
app/mailers
script
vendor
vendor/plugins
vendor/assets
vendor/assets/stylesheets
db
config
config/environments
config/locales
config/initializers
log
public
doc
test
test/integration
test/fixtures
test/functional
test/unit
test/performance
tmp
tmp/cache
tmp/cache/assets
lib
lib/tasks
lib/assets
\end{lstlisting}

Aby zainstalować konieczne do~dalszego działania wtyczki należy wyedytować plik \texttt{Gemfile} dopisując linijki z~nazwami oraz numerami wersji wtyczek.

\lstset{language=ruby, caption=Plik \texttt{Gemfile} projektu \textit{auctioneer}., basicstyle=\ttfamily\footnotesize, numbers=left, numberstyle=\footnotesize, captionpos=b, backgroundcolor=\color{LightGray}} \label{code.railsdir}
\begin{lstlisting}
source 'http://rubygems.org'

gem 'rails', '>=3.1.0'
gem 'sqlite3'

gem 'jquery-rails'
gem 'devise'
gem 'haml'
gem 'haml-rails'
gem 'formtastic'
gem 'thin'
gem 'will_paginate'
gem 'wirble', require: nil
gem 'ruby-debug19', require: nil
gem 'rspec-rails', '>= 2.6.1', group: [:development, :test]
gem 'heroku'
gem 'state_machine'

group :assets do
  gem 'sass-rails', "  ~> 3.1.0"
  gem 'coffee-rails', "~> 3.1.0"
  gem 'uglifier', '>= 1.0.3'
  gem 'therubyracer'
end

group :test do
  gem 'factory_girl_rails', '>= 1.1.0'
  gem 'cucumber-rails', '>= 1.0.2'
  gem 'capybara', '>= 1.0.1'
  gem 'database_cleaner', '>= 0.6.7'
  gem 'launchy', '>= 2.0.5'
  gem 'ruby-debug19'
  gem 'email_spec'
end
\end{lstlisting}

Po~edycji pliku \texttt{Gemfile} należy wywołać polecenie \texttt{bundle install}. Wszystkie zależności związane z~wymaganymi wtyczkami zostaną zainstalowane, a~każda zmiana w~pliku \texttt{Gemfile} wymaga ponownego uruchomienia komendy \texttt{bundle install}.

\subsubsection{Kontrola wersji Git}

W~katalogu projektu stworzonego w~poprzednim punkcie należy zainicjować system kontrolii wersji wpisując polecenie:


\texttt{git init}


W~katalogu powstanie podkatalog \texttt{.git} zawierajacy wszystkie niezbędne informacje o historii tworzonego projektu.


W~projekcie mogą znajdować się pliki, dla~których nie chcemy przechowywać historii. Mogą to~być na~przykład pliki tymczasowe, logi, pliki backup, itp. Aby uniknąć dodawania ich do~historii rozwoju projektu należy utworzyć plik \texttt{.gitignore} i~dopisać tam niechciane pliki.


Aby zapisać bieżący stan projektu w~historii należy zaznaczyć zmiany przy pomocy polecenia \texttt{git add .} oraz wykonać commit przy pomocy \texttt{git commit}. Otworzy się domyślny edytor tekstu, w którym należy podać komentarz dotyczący zmian.

\lstset{language=bash, caption=Plik \texttt{.gitignore} projektu \textit{auctioneer}., basicstyle=\ttfamily\footnotesize, captionpos=b, backgroundcolor=\color{LightGray}} \label{code.railsdir}
\begin{lstlisting}
.sass-cache/
vendor/ruby
*~
*.swp
*.swo
tags
log/*
tmp/*
lib/mysql.rb
config/*.yml
[a-zA-Z0-9\.-_]*~
misc
capybara*
coverage
config/*.sphinx.conf
config/environments/dev_local.rb
db/sphinx/*
public/system/*
public/sitemaps/*
backup
nbproject/*
features_report.html
public/services/
public/stylesheets/all.css
public/javascripts/all.js
rerun.txt
mkmf.log
.bundle/
nohup.out
*.sqlite3
.rvmrc
doc/praca_inzynierska.pdf
\end{lstlisting}

\subsubsection{Dalsza konfiguracja}

W~poprzednich krokach została przedstawiona instalacja środowiska programistycznego dla projektu napisanego w~technologii Ruby~on~Rails. Środowisko takie jest już wystarczające i można zacząć właściwą pracę nad projektem, jednakże każde z~wyżej wymienionych narzędzi można skonfigurować według własnego uznania edytując pliki konfiguracyjne.


Do~tworzenia projektu \textit{auctioneer} użyto konfiguracji dostępnej w~repozytorium GitHub: \url{https://github.com/placek/dotfiles}.


Aby zatwierdzić taką instalację należy wykonać następujące polecenie:


\texttt{git clone git://github.com/placek/dotfiles.git; cp dotfiles/* ~}

\subsection{Założenia projektu}

Zanim programista przystępuje do~pracy należy wyznaczyć szczegółowo zadania i~założenia. Wszystkie cele powinny być jasno przedstawione i~przemyślane.

\subsubsection{Założenia dotyczące projektu}

System aukcyjny powinien spełniać pewne podstawowe funkcjonalności, a~zatem powinien zapewnić możliwość:

\begin{itemize}
  \item rejestracji oraz dostępu do~konta osobom chcącym wystawiać aukcje,
  \item wystawienia aukcji oraz podania ceny minimalnej,
  \item przeglądania aukcji wraz z~ważnymi dla potencjalnych kpujących informacjami,
  \item wyszukiwania aukcji według opisu oraz podawania wyników wyszukiwania w~kolejności jaką sobie użytnownik wybierze,
  \item podbicia aukcji oraz wygrania jej na~ogólnie przyjętych zasadach.
\end{itemize}

System powinien zadbać także o~bezpieczeństwo danych użytkowników systemu oraz udostępnić możliwość administrowania danymi przez osoby uprawnione (administratorzy).

\subsubsection{Diagramy UML wyjaśniające strukturę i~zależności}

%TODO

\subsubsection{Scenariusze}

%TODO

\subsubsection{Zadania i~ich realizacja}

%TODO
