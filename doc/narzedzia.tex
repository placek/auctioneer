\section{Narzędzia użyte podczas pisania pracy}

\subsection{Kontrola pracy w~Scrum}

\begin{enumerate}
  \item \texttt{git}\footnote{\url{http://git-scm.com/}} -- system kontroli wersji bazujący na~przechowywaniu plików różnic -- tzw. plików diff.
  \item \texttt{ticgit}\footnote{\url{https://github.com/schacon/ticgit/wiki/}} -- issue tracker działajacy jako rozszerzenie dla systemu kontroli wersji git; zapisuje zmiany w~trackingu w oddzielnej gałęzi repozytorium git projektu.
  \item \texttt{Evolus Pencil}\footnote{\url{http://pencil.evolus.vn/en-US/Home.aspx}} -- narzędzie do~tworzenia mockupów oraz~szkiców założeń.
  \item \texttt{Umlet}\footnote{\url{http://www.umlet.com/}} -- narzędzie do~rysowania diagramów UML.
\end{enumerate}

\subsection{Środowisko programistyczne}

\begin{enumerate}
  \item \texttt{vim}\footnote{\url{http://www.vim.org/}} -- edytor tekstu.
  \item \texttt{RVM}\footnote{\url{https://rvm.beginrescueend.com/}} -- system kontrolii wersji języka Ruby.
  \item \texttt{rubygems}\footnote{\url{http://rubygems.org/}} -- system obsługujący bazę pluginów.
  \item \texttt{IRB} -- interaktywna konsola języka Ruby.
  \item \texttt{rspec} + \texttt{cucumber} -- patrz rozdział \ref{technologie.gemy}
  \item \texttt{Sqliteman}\footnote{\url{http://sqliteman.com/}} -- narzędzie do~zarządzania bazami danych Sqlite3.
\end{enumerate}

\subsection{Wdrożenie}

\begin{enumerate}
  \item \texttt{heroku}\footnote{\url{http://www.heroku.com/}} -- serwis pozwalający na~tworzenie deploymentu projektów napisanych w języku Ruby ,,w~chmurze''.
  \item \texttt{Nginx + thin}\footnote{Nginx: \url{http://wiki.nginx.org/NginxPl}, Thin: \url{http://code.macournoyer.com/thin/}} -- lokalne wdrażanie aplikacji przy użycu serwera HTTP \texttt{Nginx} oraz~lekkiego serwera \texttt{thin}.
\end{enumerate}
