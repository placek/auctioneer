\newpage

\section*{Wstęp}
\addcontentsline{toc}{chapter}{Wstęp}

Głównym powodem, dla którego powstała ta~praca jest próba zainteresowania programistów organizacją pracy i~metodologią Scrum. Filozofia ta, w~porównaniu ze~znanymi, prototypowymi modelami procesu powstawania oprogramowania, sprawdza się znacznie lepiej w~niewielkich zespołach projektowych. Znajomość tej metodyki (lub podobnych metodyk Agile) zwiększa szanse małym firmom programistycznym na~zaistnienie na~rynku.


Drugim, równie ważnym, powodem było zaciekawienie czytelnika technologiami i~narzędziami, takimi jak: język \textit{Ruby} oraz aplikacja szkieletowa \textit{Ruby~on~Rails}. Stosowanie tych narzędzi stanowi dobrą alternatywę dla znanych w~dziedzinie technologii, takich jak język \textit{PHP}, \textit{Java} lub \textit{.NET}.


\subsection*{Cele pracy}

Cele, jakie zamierzam osiągnąć pisząc tę~pracę są~następujące:

\begin{itemize}
  \item Przybliżenie metodologii Scrum -- zamierzam zaprezentować działanie metodologii Scrum na~przykładzie procesu tworzenia projektu systemu aukcyjnego;
  \item Opis zastosowania technologii Ruby~on~Rails w~problemie stworzenia systemu aukcyjnego -- przedstawienie procesu tworzenia aplikacji przy pomocy aplikacji szkieletowej Ruby~on~Rails;
\end{itemize}

\subsection*{Problematyka}

Praca dotyczy zagadnień inżynierii oprogramowania. Zasadniczym problemem pracy jest zaprojektowanie oraz~implementacja systemu aukcyjnego przy użyciu aplikacji szkieletowej Ruby~on~Rails.

Zakres pracy obejmuje następujące zagadnienia:

\begin{itemize}
  \item prezentacja języka Ruby oraz~aplikacji szkieletowej Ruby~on~Rails;
  \item zaprojektowanie oraz~implementacja systemu aukcyjnego przy użyciu aplikacji szkieletowej Ruby~on~Rails;
  \item przedstawienie przebiegu pracy nad projektem przy zastosowaniu się do~zasad metodologii Scrum;
\end{itemize}

\subsection*{Struktura pracy}

W~pierwszym rozdziale pracy przybliżono technologie użyte podczas tworzenia systemu aukcyjnego w~Ruby~on~Rails. Rozdział ten opisuje także narzędzia, przy pomocy których powstał ten projekt.


Rozdział drugi przybliża metodykę oraz organizację pracy nad projektem. Opisane zostały tu~konwencje stosowane przez programistów języka Ruby. Ponadto, została zaprezentowana filozofia pracy w~zespołach programistycznych, stosujących programowanie zwinne.


Rozdział trzeci opisuje szczegółowo powstały projekt -- system aukcyjny. Opisany został tu~proces instalacji i konfiguracji środowiska pracy, omówione zostały założenia projektu, przybliżono architekturę oraz przedstawiono historię pracy nad projektem. W~rozdziale tym znajduje się także podręcznik użytkownika, objaśniający istotę działania systemu.


Ostatni rozdział zawiera obserwacje poczynione podczas tworzenia systemu aukcyjnego oraz krótkie podsumowanie pracy nad projektem.
