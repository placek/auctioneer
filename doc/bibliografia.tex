\begin{thebibliography}{99}
\addcontentsline{toc}{chapter}{Bibliografia}

  \bibitem{ror} \emph{Strona domowa projektu Ruby~on~Rails} \url{http://rubyonrails.pl} (stan na~dzień \today)

  \bibitem{rspec} \emph{Narzędzie Rspec do testów jednostkowych i~funkcjonalnych} \url{http://rspec.info/} (stan na~dzień \today)

  \bibitem{cucumber} \emph{Narzędzie Cucumber pozwalające na wykonywanie testów behavioralnych} \url{http://cukes.info} (stan na~dzień \today)

  \bibitem{haml} \emph{Meta-język szablonów dokumentów XHTML} \url{http://haml-lang.com/} (stan na~dzień \today)

  \bibitem{sass} \emph{Meta-język szablonów dokumentów CSS} \url{http://sass-lang.com/} (stan na~dzień \today)

  \bibitem{devise} \emph{W~pełni konfigurowalny system autentyfikacji użytkowników dla Ruby~on~Rails} \url{https://github.com/plataformatec/devise} (stan na~dzień \today)

  \bibitem{will.paginate} \emph{Plugin paginacji stron dla Ruby~on~Rails} \url{https://github.com/mislav/will_paginate} (stan na~dzień \today)

  \bibitem{tiny.mce} \emph{Rozwinięty edytor HTML dla stron internetowych} \url{http://tinymce.moxiecode.com/} (stan na~dzień \today)

  \bibitem{sqlite3} \emph{Lekka i~szybka relacyjna baza danych SQL} \url{http://www.sqlite.org/} (stan na~dzień \today)

  \bibitem{jquery} \emph{Biblioteka programistyczna języka JavaScript} \url{http://docs.jquery.com/Main_Page} (stan na~dzień \today)

  \bibitem{git} \emph{System kontroli wersji Git} \url{http://git-scm.com} (stan na~dzień \today)

  \bibitem{ticgit} \emph{ticgit -- manager zadań projektowych dla projektów używających system kontroli wersji Git} \url{https://github.com/schacon/ticgit/wiki/} (stan na~dzień \today)

  \bibitem{ticgitweb} \emph{Webowy interfejs dla narzędzia ticgit -- ticgitweb} \url{https://github.com/schacon/ticgit/wiki/TicGitWeb} (stan na~dzień \today)

  \bibitem{zsh} \emph{Powłoka ZSH} \url{http://www.zsh.org/} (stan na~dzień \today)

  \bibitem{screen} \emph{GNU~screen} \url{http://www.gnu.org/software/screen/} (stan na~dzień \today)

  \bibitem{vim} \emph{Rozbudowany edytor tekstu Vim} \url{http://www.vim.org/} (stan na~dzień \today)

  \bibitem{rvm} \emph{System kontrolii wersji języka Ruby RVM} \url{https://rvm.beginrescueend.com/} (stan na~dzień \today)

  \bibitem{sqliteman} \emph{Sqliteman -- graficzne narzędzie do~zarządzania bazą danych Sqlite} \url{http://sqliteman.com/} (stan na~dzień \today)

  \bibitem{heroku} \emph{Heroku -- narzędzie do~wdrażania aplikacji webowych} \url{http://www.heroku.com/} (stan na~dzień \today)

  \bibitem{scrumaliance} \emph{The Scrum Framework in 30 seconds}, \url{www.scrumalliance.org/pages/what_is_scrum} (stan na~dzień \today)

  \bibitem{agile.manifesto} emph{Agile Manifesto}, \url{agilemanifesto.org/principles.html} (stan na~dzień \today)

  \bibitem{agile1} \emph{Agile and Scrum programming}, \url{www.agileprogramming.org} (stan na~dzień \today)

  \bibitem{agile2} \emph{Programowanie zwinne}, \url{http://pl.wikipedia.org/wiki/Programowanie_zwinne} (stan na~dzień \today)

  \bibitem{therubyway} \emph{Konwencja języka Ruby}, \url{https://github.com/chneukirchen/styleguide/blob/master/RUBY-STYLE} (stan na~dzień \today)

  \bibitem{html5doc} \emph{Dokumentacja standardu HTML5}, \url{http://dev.w3.org/html5/spec/Overview.html} (stan na~dzień \today)

  \bibitem{css3doc} \emph{Dokumentacja standardu HTML5}, \url{http://www.w3.org/TR/CSS/} (stan na~dzień \today)

  \bibitem{framework} \emph{Framework Design: A Role Modeling Approach}, Dirk Riehle, Swiss Federal Institute of Technology, 2000 (stan na~dzień \today)

  \bibitem{lang} \emph{Types and Programming Languages}, Benjamin C. Pierce, 2002

  \bibitem{w3} \emph{Organizacja W3 udostępniająca standardy dla witryn internetowych} \url{http://www.w3.org/} (stan na~dzień \today)

  \bibitem{cytaty} \emph{Wypowiedzi znanych programistów, wydawców literatury informatycznej dotyczące Ruby~on~Rails} \url{http://www.rubyonrails.pl/cytaty} (stan na~dzień \today)

  \bibitem{basecamp} \emph{Projekt Basecamp oparty na frameworku Ruby~on~Rails} \url{http://basecamphq.com/} (stan na~dzień \today)

  \bibitem{python} \emph{Język programowania Python} \url{http://www.python.org} (stan na~dzień \today)

  \bibitem{django} \emph{Framework Django bedący konkurencją dla Ruby~on~rails, napisany dla języka Python} \url{http://www.djangoproject.com} (stan na~dzień \today)

  \bibitem{rdoc} \emph{Narzędzie do tworzenia dokumentacji API RDoc}, \url{http://rdoc.sourceforge.net} (stan na~dzień \today)

  \bibitem{google.stats} \emph{Statystyki Google dotyczące zapytań o~znane frameworki webowe} \url{http://www.google.com/insights/search/\#cat=5\&q=Ruby\%20on\%20Rails\%2CDjango\%2CSpring\%20MVC\&cmpt=q} (stan na~dzień 20 grudnia 2010)

  \bibitem{scrum.schema} \emph{Schemat pracy w~kolejnych iteracjach metodyki Scrum}, \url{http://effectiveagiledev.com/Portals/0/800px-Scrum\_process\_svg.png} (stan na~dzień 4 stycznia 2011)

  \bibitem{ubuntu} \emph{Jedna z~najpopularniejszych dystrybucji systemu GNU~Linux -- Ubuntu} \url{http://www.ubuntu.com/} (stan na~dzień \today)

  \bibitem{pivotaltracker} \emph{Narzędzie do trackingu zadań -- PivotalTracker}, \url{www.pivotaltracker.com} (stan na~dzień \today)

  \bibitem{kernel} \emph{Strona domowa systemu Linux}, \url{http://www.kernel.org} (stan na dzień \today)

  \bibitem{css.grid} \emph{Szablony pozycjonowania CSS -- \texttt{grid layouts}}, \url{http://www.w3.org/TR/css3-grid/} (stan na~dzień \today)

  \bibitem{css.shadow} \emph{Cieniowanie obiektów w CSS -- \texttt{shadows and rounded boxes}}, \url{http://www.w3.org/TR/css3-background/} (stan na~dzień \today)

  \bibitem{zachowawcze} \emph{Programowanie zachowawcze}, \url{http://www.erlang.se/doc/programming\_rules.shtml\#HDR11} (stan na~dzień \today)

\end{thebibliography}
