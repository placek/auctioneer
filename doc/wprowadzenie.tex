\section{Wprowadzenie}

\subsubsection{Opis zastosowania technologii Ruby~on~Rails w~problemie stworzenia systemu aukcyjnego}

Technologia Ruby~on~Rails umożliwia proste tworzenie aplikacji webowych dowolnego typu. Dla zaprezentowania jej możliwości wybrałem system aukcyjny jako przykład aplikacji webowej stworzonej w~tym środowisku. Wybór ten nie jest przypadkowy -- do~tej pory nie znalazłem przykładowego systemu aukcyjnego napisanego przy użyciu aplikacji szkieletowej Ruby~on~Rails\footnote{Jedynym możliwym gotowym rozwiązaniem dla wykorzystania aplikacji webowej w~celu wystawiania aukcji/prowadzenia licytacji jest zastosowanie wtyczki % TODO nazwa i url
dla systemu CMS % TODO nazwa i url CMS - chyba spree
napisanego w~Ruby~on~Rails.}.


Pomysł jednak nie jest nowatorski -- w~sieci oraz~w~wielu pozycjach książkowych znajdują się przykłady wykonania sklepów internetowych, które są~w~budowie bardzo podobne do~systemów aukcyjnych.


System aukcyjny to~niezwykle rozwlekły i~obszerny temat. Projekt tego typu zatem może być bardzo rozbudowany. Właśnie z~tego względu zakładam, że~mój projekt nie będzie ,,dokończony''. Celem nie jest tu~wykonanie całego projektu ,,od~początku do~końca'' a~jedynie prezentacja możliwości jakie oferuje Ruby~on~Rails.

\subsubsection{Przedstawienie prototypu systemu aukcyjnego}

Wraz ze~stworzeniem prototypu systemu aukcyjnego prezentuję podstawowe rozwią\-zania dla tego rodzaju problemu. Zagadnienie stworzenia systemu aukcyjnego jest problemem typowym dla dziedziny inżynierii oprogramowania. Wymaga wybrania i~opracowania rozwiązań technicznych i~technologicznych oraz~określenia metodyki pracy nad danym zagadnieniem.


Stworzony przeze mnie prototyp jest swego rodzaju prezentacją zastosowanych w~nim technologii oraz~przykładowych rozwiązań.
