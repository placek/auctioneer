\section{Dokumentacja projektu} \label{dokumentacja}

Projekty systemów informatycznych mają to~do~siebie, że~rozwijają się niezwykle szybko. Rozwój projektu związany jest niestety z~powiększaniem objętości projektu -- zarówno merytorycznej, jak i~czysto fizycznej (ilość linijek kodu, ilość plików). Problem pojawia się, gdy projekt jest zbyt ,,duży'' żeby programista mógł rozumieć wszystko to, co~się w~nim dzieje. Rozwiązaniem dla tego problemu jest tworzenie dokumentacji.


W~projektach typu \textit{OpenSource} dokumentacja jest niezwykle ważnym czynnikiem usprawniającym pracę programistów w~nich pracujących. Dobrze napisana dokumentacja sprawia, że~nowe osoby zaczynające pracę w~projekcie mogą łatwiej i~szybciej zapoznać się ze~strukturą, działaniem oraz~organizacją projektu. Jest to~szczególnie przydatne, gdy istnieje potrzeba edycji jedynie niewielkiego fragmentu projektu. Co~więcej -- osoby zajmujące się już od~dłuższego czasu pracą w~takim projekcie nie muszą pamiętać wszelkich zagadnień z~nim związanych.


\subsection{Kod aplikacji} \label{dokumentacja.kod}

Najlepszą dokumentacją dla systemu informatycznego jest kod aplikacji. To~właśnie on~realizuje wszystkie założenia projektowe, algorytmy lub cykle pracy projektu. Prawidłowo napisany kod może powiedzieć więcej niż nie jedna dokumentacja -- tu~dowiadujemy się, jak naprawdę działa interesujący nas moduł i~mamy pewność, że~nie padniemy ofiarą nieporozumień. Pod pojęciem ,,prawidłowo napisany kod'' rozumie się spełnienie następujących założeń:

\begin{itemize}
  \item trzymanie się konwencji określającej osnowę dokumentu, zawierającego kod,
  \item stosowanie zrozumiałych nazw dla wszelkich struktur (nazwy klas, metod, zmiennych),
  \item tworzenie krótkich i~treściwych fragmentów kodu oraz~rozbijanie większych partii na~mniejsze.
\end{itemize}

Wszystkie te kroki mają na celu sprawienie, że~kod stanie się bardziej czytelny. Poniżej omówiono te~trzy kroki nieco dokładniej w~zastosowaniu dla języka Ruby.

\subsubsection{Konwencja} \label{dokumentacja.konwencja}

Język Ruby ze~względu na~swoją składnię jest niezwykle czytelny, a~co~za~tym idzie, kod w~nim napisany jest łatwy w~zrozumieniu. Składnia Rubiego przypomina pseudokod:

  \lstset{language=Ruby, caption=Przykład prostej składni języka Ruby -- algorytm DFS., basicstyle=\ttfamily\footnotesize, numbers=left, numberstyle=\footnotesize, captionpos=b, backgroundcolor=\color{LightGray}} \label{code.simpleruby}
  \begin{lstlisting}
  def dfs node, value, queue
    return false if node.nil?
    return true if node.data == value
    queue.push node.right_neighbor unless node.right_neighbor.nil?
    queue.push node.left_neighbor unless node.left_neighbor.nil?
    dfs queue.pop, value, queue
  end
  \end{lstlisting}

Niestety -- sama składnia nie jest najważniejsza w~dokumentowaniu projektów. Powyższy przykład można przecież przepisać w~inny sposób:

  \lstset{language=Ruby, caption=Algorytm DFS z~zastosowaniem braku konwencji formatu., basicstyle=\ttfamily\footnotesize, numbers=left, numberstyle=\footnotesize, captionpos=b, backgroundcolor=\color{LightGray}}
  \begin{lstlisting}
  def dfs node, value, queue
  return false if node.nil?; if node.data == value
  return true
  end; queue.push node.right_neighbor unless \\
    node.right_neighbor.nil?
  unless node.left_neighbor.nil?
  queue.push node.left_neighbor; end
  dfs queue.pop, value, queue
  end
  \end{lstlisting}

Przykład ten jest mniej czytelny, a~co~za~tym idzie, trudniej jest dowiedzieć się, za~co~dany fragment kodu jest odpowiedzialny. W~celu uzyskania ,,przejrzystości'' kodu stosuje się konwencje zapisu -- ogólnie przyjęte zasady mówiące o~tym jak powinien wyglądać kod i~jak ten kod formatować. Język Ruby posiada swoją konwencję o~nazwie \textit{The Ruby Style} \cite{therubyway}, która określona jest następująco:

\begin{enumerate}
  \item Formatowanie:
  \begin{enumerate}
    \item Używaj zawsze kodowania ASCII lub UTF8.
    \item Używaj dwóch spacji jako wcięć (nigdy tabulatur).
    \item Staraj się kończyć wiersz w~stylu używanym w~systemach Unix (LF -- 0x0A).
    \item Używaj spacji przed i~po~operatorach, po~przecinkach, po~dwukropkach, po~średnikach, przed i~po~\{ oraz~po~\}.
    \item Nie używaj spacji po~( oraz~[. Nie używaj spacji przed~] oraz~).
    \item Używaj dwóch spacji przed modyfikatorem warunku (\texttt{if}/\texttt{unless}/\texttt{while}/\texttt{until}/\texttt{resque}).
    \item Wcięcie dla słowa kluczowego \texttt{when} powinno być tak głębokie, jak dla \texttt{case}.
    \item Użyj pustego wiersza przed zwracaną wartością w~metodzie (chyba, że~ma~ona tylko jeden wiersz), a~także przed słowem kluczowym \texttt{def}
    \item Używaj RDoc do~dokumentacji API. Nie wstawiaj pustego wiersza pomiędzy komentarz a~komentowany blok.
    \item Użyj pustego wiersza do~podzielenia dużych metod na~logiczne fragmenty.
    \item Staraj się, aby~każdy wiersz miał mniej niż 80 znaków.
    \item Unikaj białych znaków na~końcu wiersza.
  \end{enumerate}
  \item Składnia:
  \begin{enumerate}
    \item Użyj \texttt{def} z~ nawiasami, gdy są~podane argumenty.
    \item Nie używaj \texttt{for}, chyba, że~robisz to~celowo.
    \item Nie używaj \texttt{then}.
    \item Użyj ,,\texttt{when x;} \ldots'' dla jednolinijkowego wyrażenia \texttt{case}.
    \item Użyj \texttt{\&\&}/\texttt{||} dla wyrażeń boolowskich, \texttt{and}/\texttt{or} do~kontroli przepływu.
    \item Unikaj wielolinikowej instrukcji \texttt{?:}, użyj \texttt{if}.
    \item Zaniechaj użycia nawiasów przy wywołaniu metod, ale użyj ich podczas wywołania ,,funkcji'' (na~przykład~gdy używasz zwracanej wartości w~tym samym wierszu)
    \item Używaj raczej \mbox{\texttt{\{} \ldots \texttt{\}}} niż \mbox{\texttt{do} \ldots \texttt{end}}. Wielolinijkowe bloki \mbox{\texttt{\{} \ldots \texttt{\}}} są~w~porządku: używając \texttt{\}} na końcu bloku wiemy, że~kończy się blok a~nie instrukcja \texttt{if}/\texttt{while}/\ldots. Używaj \mbox{\texttt{do} \ldots \texttt{end}} do~kontroli przepływu (na~przykład~zadania \texttt{rake}, bloki \texttt{sinatra}).
    \item Unikaj używania słowa kluczowego \texttt{return} jeśli nie jest potrzebne.
    \item Unikaj kontynuacji linii (\texttt{$ \backslash $}) jeśli nie musisz.
    \item Używanie zwracanej wartości przez operator \texttt{=} jest na miejscu.
    \item Używaj operatora \texttt{||=}.
    \item Używaj wyrażeń regularnych typu ,,non-OO''.  Nie bój się używać \texttt{=~}, \texttt{\$0-9}, \texttt{\$~}, \texttt{\$`} oraz~\texttt{\$'} jeśli potrzebujesz.
  \end{enumerate}
  \item Nazewnictwo:
  \begin{enumerate}
    \item Używaj \textit{snake\_case} jako stylu nazywania metod.
    \item Używaj \textit{CamelCase} jako stylu nazywania klas i~modułów (pozostaw akronimy takie jak HTTP, RFC, XML z~wielkich liter).
    \item Używaj SCREAMING\_SNAKE\_CASE jako stylu nazywania stałych.
    \item Długość nazwy zwykle określa kontekst wykorzystania. Używaj jednoliterowych zmiennych jako parametrów bloków/metod według tego schematu:\\
      \texttt{a}, \texttt{b}, \texttt{c}, \texttt{o}: dowolny obiekt;\\
      \texttt{d}: katalog;\\
      \texttt{e}: element (klasy Emumerable oraz~pochodnych);\\
      \texttt{ex}: wyjątek;\\
      \texttt{f}: plik:\\
      \texttt{i}, \texttt{j}: indeksy;\\
      \texttt{k}: klucz tablicy asocjacyjnej;\\
      \texttt{m}: metoda;\\
      \texttt{r}: zwracana wartość krótkich metod;\\
      \texttt{s}: tekst;\\
      \texttt{v}: wartość elementu tablicy asocjacyjnej;\\
      \texttt{x}, \texttt{y}, \texttt{z}: liczby;\\
      Ponadto, pierwsza litera klasy obiektu może posłużyć za~nazwę takiej zmiennej.

    \item Używaj nazw zaczynających się od~\texttt{\_} dla nieużywanych zmiennych.
    \item Używając \texttt{inject} dla krótkich bloków nazywaj argumenty \texttt{|a, e|} (od:~akumulator, element).
    \item Definiując operatory dwuargumentowe, nazywaj argument jako ,,other''.
    \item Używaj raczej \texttt{map} niż \texttt{collect}, \texttt{find} niż \texttt{detect}, \texttt{find\_all} niż \texttt{select} oraz~\texttt{size} niż \texttt{length}.
  \end{enumerate}
  \item Komentarze:
  \begin{enumerate}
    \item Komentarze dłuższe niż słowo rozpoczynają się z~wielkiej litery i~używane są zasady interpunkcji.
    \item Używaj dwóch spacji po każdej kropce w komentarzu.
    \item Unikaj niepotrzebnych komentarzy.
  \end{enumerate}
  \item Pozostałe:
  \begin{enumerate}
    \item Pisz kod przyjazny dla opcji \texttt{ruby -w}.
    \item Unikaj tablic asocjacyjnych jako opcjonalnych parametrów.
    \item Unikaj długich metod.
    \item Unikaj długich list parametrów.
    \item Używaj konstrukcji \texttt{def self.}\textit{metoda} dla zdefiniowania metod singletonów.
    \item Staraj się rozwijać funkcjonalność standardowych metod.
    \item Unikaj \texttt{alias} -- używaj \texttt{alias\_method}.
    \item Używaj \texttt{OptionParser} do~parsowania skomplikowanych opcji wejścia konsoli a~\texttt{ruby -s} dla rozwiązań trywialnych.
    \item Staraj się zachować zgodność wielu wersji interpretera.
    \item Unikaj zbędnego metaprogramowania.
  \end{enumerate}
  \item Ogólne zasady:
  \begin{enumerate}
    \item Programuj w sposób funkcjonalny.
    \item Nie wydziwiaj z~używaniem argumentów metod -- chyba, że~wiesz co robisz.
    \item Nie zmieniaj funkcjonalności standardowych bibliotek pisząc własne.
    \item Nie programuj zachowawczo\footnote{Patrz: \url{http://www.erlang.se/doc/programming\_rules.shtml\#HDR11}}.
    \item Postaraj się zachować prostotę kodu.
    \item Nie przesadź z~projektowaniem.
    \item Ale także nie pozostaw swojej pracy niezaprojektowanej.
    \item Unikaj błędów.
    \item Poczytaj o innych konwencjach, aby móc rozwinąć tę.
    \item Bądź konsekwentny.
    \item Używaj prostych rozwiązań.
  \end{enumerate}
\end{enumerate}

Stosowanie tej konwencji zapewni, że~kod napisany przez nas będzie zgodny z~ogólnym standardem.

\subsubsection{Nazewnictwo w~kodzie} \label{dokumentacja.nazewnictwo}

Aby zrozumieć istotę działania projektu należy zrozumieć mechanizmy i~algorytmy rządzące jego logiką. Zakładając, że~chcemy dobrze odokumentować projekt, trzeba tak napisać kod, by~był zrozumiały -- jak pseudokod opisujący algorytm. W~tym celu wystarczy, że~wszelkie obiekty w~ kodzie, nazwiemy tak, by~ich użycie w~kontekscie było zrozumiałe a~ich rola jasna. W~trakcie pisania kodu najtrudniejszą kwestią jest nazywanie obiektów, a~nie -- jak to~się powszechnie uznaje -- rozwiązanie logiki systemu.

\subsection{Komentarze} \label{dokumentacja.komentarze}

Komentarze w~kodzie to~dobrzy sposób na~wyjaśnienie zasadności użycia algorytmów, bądź struktury API poszczególnych elementów kodu. Stosowanie komentarzy pozwala także wyróżnić, co~ważniejsze dla ,,czytelnika'' elementy kodu.

\subsubsection{RDoc} \label{dokumentacja.rdoc}

Istnieje wiele konwencji dotyczących~formy komentarzy. Format komentarzy ma~znaczenie nie tylko podczas czytania kodu -- może on~posłużyć pośrednio do~wygenerowania pełnej dokumentacji API przy użyciu odpowiednich narzędzi.


Komentarze w~języku Ruby mogą być dwojakiego formatu:

  \lstset{language=Ruby, caption=Komentarze w~języku Ruby., basicstyle=\ttfamily\footnotesize, numbers=left, numberstyle=\footnotesize, captionpos=b, backgroundcolor=\color{LightGray}}
  \begin{lstlisting}
    # Komentarz jednolinijkowy rozpoczyna sie
    # od znaku krzyzyka.

    =begin
      Blok komentarza.
      Taki blok rozpoczyna sie od znacznika "=begin"
      a konczy na znaczniku "=end".
      Komentarze jednolinijkowe sa jednak czesciej
      uzywane.
    =end
  \end{lstlisting}


Narzędzie \textit{RDoc}\footnote{Patrz: \url{rdoc.sourceforge.net}} to~narzędzie konsolowe generujące dokumentację API w~zadanym formacie (standardowo jest to~HTML) na~podstawie kodu oraz~komentarzy w~nim zawartych. Aby wygenerować taką dokumentację wystarczy wpisać w~konsoli:

\mbox{\texttt{\$ rdoc <opcje> [plik...] }}

\textit{RDoc} wygeneruje czytelną dokumentację nawet bez komentarzy w~kodzie.

\subsection{Testy} \label{dokumentacja.testy}

Testy to~dodatkowe aplikacje sprawdzające poprawność logiki projektu. Zasadniczą ich funkcją jest dowiedzenie, że~dana funkcjonalność została zaimplementowana poprawnie.


Testować w~aplikacji można wszystko, jednak dobrze jest ustalić co~chce się uzyskać poprzez napisanie testów. Biorąc pod uwagę powyższe kryterium testy dzieli się na:

\begin{enumerate}
 \item \textit{testy jednostkowe} -- sprawdzają poprawność modułów ,,silnika'' aplikacji. Za~pomocą testów jednostkowych testuje się zwykle klasy, metody, stany maszyny, poprawność algorytmów, wejścia i~wyjścia strumieni itp.
 \item \textit{testy behawioralne} -- odpowiadają na~pytanie: ,,co~się stanie, gdy zrobimy \ldots''. Testują one zachowanie aplikacji -- reakcję na~żądania, efekty działania zdarzeń (takich jak, na przykład, wypełnienie formularza).
\end{enumerate}

\subsubsection{Testy jednostkowe RSpec} \label{dokumentacja.rspec}

\textit{RSpec}\footnote{Patrz: \url{rspec.info}} jest narzędziem do~tworzenia testów jednostkowych dla aplikacji napisanych w~języku Ruby. Jego składnia jest prosta, a~testy w~nim napisane -- czytelne:

  \lstset{language=Ruby, caption=Test RSpec testujący klasę \texttt{Array}., basicstyle=\ttfamily\footnotesize, numbers=left, numberstyle=\footnotesize, captionpos=b, backgroundcolor=\color{LightGray}} \label{code.prostytest}
  \begin{lstlisting}
  describe Array do
    describe "#push" do
      it "puts a value at the end of array" do
        array = Array.new
        value = "Some value"
        array.push value
        array.last.should == value
      end
    end
  end
  \end{lstlisting}

W~pierwszej kolejności podaje się opis testu -- krótką informację o~tym co~testujemy i~jakie są oczekiwania względem testowanego obiektu. Wewnątrz bloku testu realizuje się przypadek użycia obiektu. W~każdym momencie można sprawdzić, czy interesujący nas stan obiektu jest zgodny z~naszymi oczekiwaniami. W~tym celu używa się metod \texttt{should} oraz~\texttt{should\_not}.


Testy uruchamiamy podając w~konsoli polecenie \texttt{rspec} oraz~plik z~napisanymi testami:

\mbox{\texttt{\$ rspec nazwa\_pliku\_testu.rb}}

Trzeba pamiętać, że~testy powinny mieć dostęp do~logiki projektu -- na~przykład~poprzez użycie instrukcji \texttt{require}. Efektem działania testu z~przykładu \ref{code.prostytest} będzie:

  \lstset{language=bash, caption=Efekt uruchomienia testu jednostkowego z~poprzedniego przykładu., basicstyle=\ttfamily\footnotesize, numbers=left, numberstyle=\footnotesize, captionpos=b, backgroundcolor=\color{LightGray}}
  \begin{lstlisting}
.

Finished in 0.00258 seconds
1 example, 0 failures
  \end{lstlisting}

Aby zrozumieć, co~stało się podczas testowania, można zmienić format wyjścia na~bardziej przyjazny:

  \lstset{language=bash, caption=Efekt uruchomienia testu jednostkowego z~opcją \texttt{----format d}., basicstyle=\ttfamily\footnotesize, numbers=left, numberstyle=\footnotesize, captionpos=b, backgroundcolor=\color{LightGray}}
  \begin{lstlisting}
Array
  #push
    puts a value at the end of array

Finished in 0.0029 seconds
1 example, 0 failures
\end{lstlisting}

Jak widać \texttt{rspec} poinformował o~tym, które testy ,,przeszły'' -- czyli, które z~testowanych przez funkcjonalności spełniły oczekiwania.

\subsubsection{RSpec jako dokumentacja}

Przykład \ref{code.prostytest} mówi dość dużo o~sposobie użycia elementów logiki projektu. Testy jednostkowe są~tak naprawdę przypadkami użycia tychże elementów. Co~więcej -- przypadki te~realizują się zgodnie z~założeniami twórcy takiego projektu (uruchomione testy ,,przechodzą''). Oznacza to, że~są one dobrą dokumentacją działania i~sposobu użycia poszczególnych obiektów logiki projektu.

\subsubsection{Dokumentacja poprzez testy behawioralne} \label{dokumentacja.cucumber}

Skoro testy jednostkowe są~swego rodzaju dokumentacją, to~także testy behawioralne mogą nią być. Oczywiście ze~względu na~inny cel testy behawioralne będą dokumentować projekt z~innej perspektywy. Testy te~określają zachowanie aplikacji, a~zatem kwestii ściśle związanej z~użytkowaniem. W~praktyce testy behawioralne są~stosowane dla określenia wymagań ze~strony klienta -- zleceniodawcy projektu.


Istnieje kilka narzędzi, przy pomocy których można napisać i~przeprowadzić testy behawioralne. Można do~tego użyć \texttt{rspec} oraz~\texttt{Test::Unit} (czyli narzędzi wykorzystywanych przy pisaniu testów jednostkowych). Dla dobrej czytelności oraz~eleganckiego formatu użyto narzędzia \textit{Cucumber}\footnote{Patrz: \url{cukes.info}}.
