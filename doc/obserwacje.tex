Aplikacja szkieletowa Ruby~on~Rails ma~wiele zalet pozwalających na~pracę przy pomocy metodyki programowania zwinnego. Struktura aplikacji opartych na~frameworku Ruby~on~Rails jest przejrzysta, jasna i~zorganizowana. Kod takiej aplikacji jest czytelny, a~do~jego zrozumienia wystarczy znajomość języka angielskiego.


Ponadto ściśle wyodrębnione warstwy aplikacji pozwalają na~logiczny podział zadań poszczególnych komponentów oraz na~uniknięcie nadmiarowych bądź nieoczekiwanych operacji. Zapewnia to~bezpieczeństwo oraz dobrą kontrolę nad realizacją funkcjonalności.


Ruby~on~Rails posiada szereg wtyczek, dodatków, które realizują mniej lub bardziej skomplikowane działania. Można nimi w~prosty sposób zarządzać, co~sprawia, że~praca nad aplikacją jest prostsza i~bardziej efektywna.

\section{Obserwacje}

Podczas realizacji pracy poczyniono następujące obserwacje:

\begin{itemize}
  \item Język \textit{Ruby} ze względu na~swoją naturę umożliwia swobodne tworzenie dowolnego typu aplikacji. Możliwość wyboru pomiędzy programowaniem strukturalnym, obiektowym, funkcyjnym lub nawet metaprogramowaniem nie ogranicza programisty, a~wręcz pozwala mu~na~szybszą implementację założeń projektowych bez potrzeby zastanawiania się nad szczegółami.
  \item System zarządzania wtyczkami \textit{RubyGems} wraz z~pomocą narzędzia \texttt{bundler} zapewnia doskonałą kontrolę nad zależnościami projektu bez potrzeby troski o~instalację wtyczek.
  \item Ścisły podział aplikacji Ruby~on~Rails na~warstwy (model MVC) pozwala określić dokładnie zadania i~cele poszczególnych komponentów. Umożliwia to~lepszą kontrolę nad realizacją funkcjonalności oraz eliminację błędów mogących prowadzić do~poważnych usterek bezpieczeństwa danych biznesowych.
  \item Możliwość hierarchizacji komponentów warstwy logiki (kontrolerów) pomaga w~dobrej organizacji zakresu działań i~akcji przeprowadzanych przez użytkowników aplikacji na~danych.
  \item Podział warstwowy pozwala także zwiększyć czytelność kodu, co~znacznie przyspiesza i~usprawnia pracę nad nim.
  \item Interesujący system tworzenia kwerend modułu \texttt{ActiveRecord} pozwala na~szybkie i~czytelne konstruowanie zapytań bazy danych bez potrzeby konstruowania skomplikowanych kwerend w~języku \textit{SQL}. Pozwala to~także na~wykonywanie operacji na~bazie danych niezależnie od~systemu zarządzania bazą danych (DBMS).
  \item Testy jednostkowe i~behawioralne pozwalają szybko sprawdzić poprawność działania poszczególnych elementów aplikacji bez potrzeby jej uruchamiania, czy wdrażania.
  \item Narzędzie \texttt{cucumber} wspomaga współpracę z~potencjalnym klientem: scenariusze testujące funkcjonalność mogą być pisane przez programistę i~przez klienta. W~zasadzie tego typu scenariusze wraz z~szkicami (mockupami) stanowią pełną dokumentację przypadków użycia (a zatem funkcjonalności, założeń projektu).
  \item Symulacja pracy w~metodyce Scrum wyłania jasno określone cele, przez co~zadania są~szczegółowe oraz brak w~nich niedopowiedzeń.
  \item System kontroli wersji pozwala na~sprawną pracę nad projektem w~przypadkach częstych zmian w~projekcie. Jest on~dobrym narzędziem, wspomagającym pracę w~metodyce programowania zwinnego.
\end{itemize}
