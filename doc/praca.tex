\documentclass[12pt]{report}
\usepackage[T1]{fontenc}
\usepackage[utf8]{inputenc}
\usepackage{graphicx}
\usepackage{amsmath,amssymb,amsfonts}
\usepackage{txfonts}
\usepackage[polish]{babel}
\renewcommand{\chaptername}{Rozdział}
\renewcommand{\contentsname}{Spis treści}
\renewcommand{\figurename}{Rys.}
\renewcommand{\tablename}{Tab.}
\renewcommand{\listfigurename}{Spis rysunków}
\renewcommand{\listtablename}{Spis tabel}
\renewcommand{\bibname}{Bibliografia}

\pagestyle{headings}

\setlength{\textwidth}{14cm}
\setlength{\textheight}{20cm}

\newtheorem{definition}{Definicja}
\newtheorem{example}{Przykład}[chapter]
\newtheorem{corollary}{Wniosek}[chapter]


\begin{document}
\tableofcontents

\chapter{Wstęp} \label{rozdz.wstep}
{

\section{Problematyka i zakres pracy}
{

}

\section{Cele pracy}
{
\subsubsection{Przybliżenie (popularyzacja) metodologii scrum}
{

}
\subsubsection{Przybliżenie (popularyzacja) technologii Ruby on Rails}
{

}\subsubsection{Przybliżenie (popularyzacja) }
{

}
\subsubsection{Propozycja rozwiązania problemu....}
{

}
\subsubsection{Opis zastosowania technologii Ruby on Rails w problemie stworzenia systemu aukcyjnego}
{

}
\subsubsection{Przedstawienie prototypu systemu aukcyjnego}
{

}
\subsubsection{Ocena możliwości wdrożenia proponowanych rozwiązań, ich wartość praktyczna, lokalne i globalne możliwości zastosowania}
{

}
}

\section{Metoda badawcza}
{
\begin{itemize}
\item Studia literaturowe
\item Analiza budowy i działania istniejących produktów
\item Projektowanie i prototypowanie nowatorskich rozwiązań 
\item Obliczenia i ........
\end{itemize}
}

\section{Przegląd literatury w dziedzinie}
Rozszerzyć odpowiedni podpunkt z metody badawczej, np. wg podziału:
\subsubsection{Źródła książkowe polskojęzyczne i tłumaczenia}
\subsubsection{Źródła książkowe obcojęzyczne}
\subsubsection{Artykuły naukowe, raporty z badań, komunikaty konferencyjne,
dokumentacje techniczne, manuale, instrukcje}
\subsubsection{Źródła elektroniczne}


\section{Układ pracy}
Tematem pracy jest: ....., zaś za główny cel przyjęto ...... . \\
Rozdział \label{rozdz.wstep} zawiera wstęp i cele pracy. W rozdziale drugim
opisano/...... w Rozdziale 3. zawarto............ Rozdział 4. przedstawia..... \\
W podsumowaniu pracy przedstawiono..........................., z czego wynika,
że ................  \\
Najważniejszym wnioskiem/wynikiem/rezultatem pracy jest..................\\ {\bf wyraźnie określić
CO TO JEST}. \\

{\bf cały podrozdział ok. 1 strony}.




\chapter{Tytuł części teoretycznej} \label{etykietarozdzialu2}

\section{Dokumentacja projektu} \label{dokumentacja}

Projekty systemów informatycznych mają to~do~siebie, że~rozwijają się niezwykle szybko. Rozwój projektu związany jest niestety z~powiększaniem objętości projektu -- zarówno merytorycznej, jak i~czysto fizycznej (ilość linijek kodu, ilość plików). Problem pojawia się, gdy projekt jest zbyt ,,duży'', żeby programista mógł rozumieć wszystko to, co~się w~nim dzieje. Rozwiązaniem dla tego problemu jest tworzenie dokumentacji.


W~projektach typu \textit{OpenSource} dokumentacja jest niezwykle ważnym czynnikiem usprawniającym pracę programistów w~nich pracujących. Dobrze napisana dokumentacja sprawia, że~nowe osoby zaczynające pracę w~projekcie mogą łatwiej i~szybciej zapoznać się ze~strukturą, działaniem oraz~organizacją projektu. Jest to~szczególnie przydatne, gdy istnieje potrzeba edycji jedynie niewielkiego fragmentu projektu. Co~więcej -- osoby zajmujące się już od~dłuższego czasu pracą w~takim projekcie nie muszą pamiętać wszelkich zagadnień z~nim związanych.


\subsection{Kod aplikacji} \label{dokumentacja.kod}

Najlepszą dokumentacją dla systemu informatycznego jest kod aplikacji. To~właśnie on~realizuje wszystkie założenia projektowe, algorytmy lub cykle pracy projektu. Prawidłowo napisany kod może powiedzieć więcej niż nie jedna dokumentacja -- tu~dowiadujemy się, jak naprawdę działa interesujący nas moduł i~mamy pewność, że~nie padniemy ofiarą nieporozumień. Pod pojęciem ,,prawidłowo napisany kod'' rozumie się spełnienie następujących założeń:

\begin{itemize}
  \item ,,trzymanie się'' konwencji określającej osnowę dokumentu, zawierającego kod,
  \item stosowanie zrozumiałych nazw dla wszelkich struktur (nazwy klas, metod, zmiennych),
  \item tworzenie krótkich i~treściwych fragmentów kodu oraz~rozbijanie większych partii kodu na~mniejsze.
\end{itemize}

Wszystkie te kroki mają na celu sprawienie, że~kod stanie się bardziej czytelny. Poniżej omówiono te~trzy kroki nieco dokładniej w~zastosowaniu dla języka Ruby.

\subsubsection{Konwencja} \label{dokumentacja.konwencja}

Język Ruby ze~względu na~swoją składnię jest niezwykle czytelny, a~co~za~tym idzie, kod w~nim napisany jest łatwy w~zrozumieniu. Składnia Rubiego przypomina pseudokod (listing \ref{code.pseudocode}).

  \lstset{language=Ruby, caption=Przykład prostej składni języka Ruby -- algorytm DFS, basicstyle=\ttfamily\footnotesize, numbers=left, numberstyle=\footnotesize, captionpos=b, backgroundcolor=\color{LightGray}}
  \begin{lstlisting}[label={code.pseudocode}]
  def dfs node, value, queue
    return false if node.nil?
    return true if node.data == value
    queue.push node.right_neighbor unless node.right_neighbor.nil?
    queue.push node.left_neighbor unless node.left_neighbor.nil?
    dfs queue.pop, value, queue
  end
  \end{lstlisting}

Niestety sama składnia nie jest najważniejsza w~dokumentowaniu projektów. Powyższy przykład można przecież przepisać w~inny sposób (listing \ref{code.unclear}

  \lstset{language=Ruby, caption=Algorytm DFS z~zastosowaniem braku konwencji formatu, basicstyle=\ttfamily\footnotesize, numbers=left, numberstyle=\footnotesize, captionpos=b, backgroundcolor=\color{LightGray}}
  \begin{lstlisting}[label={code.unclear}]
  def dfs node, value, queue
  return false if node.nil?; if node.data == value
  return true
  end; queue.push node.right_neighbor unless \\
    node.right_neighbor.nil?
  unless node.left_neighbor.nil?
  queue.push node.left_neighbor; end
  dfs queue.pop, value, queue
  end
  \end{lstlisting}

Przykład ten jest mniej czytelny, a~co~za~tym idzie, trudniej jest dowiedzieć się, za~co~dany fragment kodu jest odpowiedzialny. W~celu uzyskania ,,przejrzystości'' kodu stosuje się konwencje zapisu -- ogólnie przyjęte zasady mówiące o~tym, jak powinien wyglądać kod i~jak ten kod formatować. Język Ruby posiada swoją konwencję o~nazwie \textit{The Ruby Style} \cite{therubyway}, która określona jest następująco:

\begin{enumerate}
  \item Formatowanie:
  \begin{enumerate}
    \item Używaj zawsze kodowania ASCII lub UTF8.
    \item Używaj dwóch spacji jako wcięć (nigdy tabulatur).
    \item Staraj się kończyć wiersz w~stylu używanym w~systemach Unix (LF -- 0x0A).
    \item Używaj spacji przed i~po~operatorach, po~przecinkach, po~dwukropkach, po~średnikach, przed i~po~\{ oraz~po~\}.
    \item Nie używaj spacji po~( oraz~[. Nie używaj spacji przed~] oraz~).
    \item Używaj dwóch spacji przed modyfikatorem warunku (\texttt{if}/\texttt{unless}/\texttt{while}/\texttt{until}/\texttt{resque}).
    \item Wcięcie dla słowa kluczowego \texttt{when} powinno być tak głębokie, jak dla \texttt{case}.
    \item Użyj pustego wiersza przed zwracaną wartością w~metodzie (chyba, że~ma~ona tylko jeden wiersz), a~także przed słowem kluczowym \texttt{def}
    \item Używaj \textit{RDoc} do~dokumentacji API. Nie wstawiaj pustego wiersza pomiędzy komentarz a~komentowany blok.
    \item Użyj pustego wiersza do~podzielenia dużych metod na~logiczne fragmenty.
    \item Staraj się, aby~każdy wiersz miał mniej niż 80 znaków.
    \item Unikaj białych znaków na~końcu wiersza.
  \end{enumerate}
  \item Składnia:
  \begin{enumerate}
    \item Użyj \texttt{def} z~ nawiasami, gdy są~podane argumenty.
    \item Nie używaj \texttt{for}, chyba, że~robisz to~celowo.
    \item Nie używaj \texttt{then}.
    \item Użyj ,,\texttt{when x;} \ldots'' dla jednolinijkowego wyrażenia \texttt{case}.
    \item Użyj \texttt{\&\&}/\texttt{||} dla wyrażeń boolowskich, \texttt{and}/\texttt{or} do~kontroli przepływu.
    \item Unikaj wielolinikowej instrukcji \texttt{?:}, użyj \texttt{if}.
    \item Zaniechaj użycia nawiasów przy wywołaniu metod, ale użyj ich podczas wywołania ,,funkcji'' (na~przykład~gdy używasz zwracanej wartości w~tym samym wierszu).
    \item Używaj raczej \mbox{\texttt{\{} \ldots \texttt{\}}} niż \mbox{\texttt{do} \ldots \texttt{end}}. Wielolinijkowe bloki \mbox{\texttt{\{} \ldots \texttt{\}}} są~poprawne: używając \texttt{\}} na końcu bloku wiemy, że~kończy się blok a~nie instrukcja \texttt{if}/\texttt{while}/\ldots. Używaj \mbox{\texttt{do} \ldots \texttt{end}} do~kontroli przepływu (na~przykład~zadania \texttt{rake}, bloki \texttt{sinatra}).
    \item Unikaj używania słowa kluczowego \texttt{return}, jeśli nie jest potrzebne.
    \item Unikaj kontynuacji linii (\texttt{$ \backslash $}), jeśli nie musisz.
    \item Używanie zwracanej wartości przez operator \texttt{=} jest na miejscu.
    \item Używaj operatora \texttt{||=}.
    \item Używaj wyrażeń regularnych typu ,,non-OO''.  Nie bój się używać \texttt{=~}, \texttt{\$0-9}, \texttt{\$~}, \texttt{\$`} oraz~\texttt{\$'} jeśli potrzebujesz.
  \end{enumerate}
  \item Nazewnictwo:
  \begin{enumerate}
    \item Używaj \textit{snake\_case} jako stylu nazywania metod.
    \item Używaj \textit{CamelCase} jako stylu nazywania klas i~modułów (pozostaw akronimy takie jak HTTP, RFC, XML z~wielkich liter).
    \item Używaj SCREAMING\_SNAKE\_CASE jako stylu nazywania stałych.
    \item Długość nazwy zwykle określa kontekst wykorzystania. Używaj jednoliterowych zmiennych jako parametrów bloków/metod według schematu:\\
      \texttt{a}, \texttt{b}, \texttt{c}, \texttt{o}: dowolny obiekt;\\
      \texttt{d}: katalog;\\
      \texttt{e}: element (klasy \texttt{Emumerable} oraz~pochodnych);\\
      \texttt{ex}: wyjątek;\\
      \texttt{f}: plik:\\
      \texttt{i}, \texttt{j}: indeksy;\\
      \texttt{k}: klucz tablicy asocjacyjnej;\\
      \texttt{m}: metoda;\\
      \texttt{r}: zwracana wartość krótkich metod;\\
      \texttt{s}: tekst;\\
      \texttt{v}: wartość elementu tablicy asocjacyjnej;\\
      \texttt{x}, \texttt{y}, \texttt{z}: liczby;\\
      Ponadto, pierwsza litera klasy obiektu może posłużyć za~nazwę takiej zmiennej.

    \item Używaj nazw zaczynających się od~\texttt{\_} dla nieużywanych zmiennych.
    \item Używając \texttt{inject} dla krótkich bloków nazywaj argumenty \texttt{|a, e|} (akumulator, element).
    \item Definiując operatory dwuargumentowe, nazywaj argument jako ,,other''.
    \item Używaj raczej \texttt{map} niż \texttt{collect}, \texttt{find} niż \texttt{detect}, \texttt{find\_all} niż \texttt{select} oraz~\texttt{size} niż \texttt{length}.
  \end{enumerate}
  \item Komentarze:
  \begin{enumerate}
    \item Komentarze dłuższe niż słowo rozpoczynają się z~wielkiej litery i~używane są zasady interpunkcji.
    \item Używaj dwóch spacji po każdej kropce w komentarzu.
    \item Unikaj niepotrzebnych komentarzy.
  \end{enumerate}
  \item Pozostałe:
  \begin{enumerate}
    \item Pisz kod przyjazny dla opcji \texttt{ruby -w}.
    \item Unikaj tablic asocjacyjnych jako opcjonalnych parametrów.
    \item Unikaj długich metod.
    \item Unikaj długich list parametrów.
    \item Używaj konstrukcji \texttt{def self.}\textit{metoda} dla zdefiniowania metod singletonów.
    \item Staraj się rozwijać funkcjonalność standardowych metod.
    \item Unikaj \texttt{alias} -- używaj \texttt{alias\_method}.
    \item Używaj \texttt{OptionParser} do~parsowania skomplikowanych opcji wejścia konsoli a~\texttt{ruby -s} dla rozwiązań trywialnych.
    \item Staraj się zachować zgodność wielu wersji interpretera.
    \item Unikaj zbędnego metaprogramowania.
  \end{enumerate}
  \item Ogólne zasady:
  \begin{enumerate}
    \item Programuj w~sposób funkcjonalny.
    \item Nie wydziwiaj z~używaniem argumentów metod -- chyba, że~wiesz co robisz.
    \item Nie zmieniaj funkcjonalności standardowych bibliotek pisząc własne.
    \item Nie programuj zachowawczo \cite{zachowawcze}.
    \item Postaraj się zachować prostotę kodu.
    \item Nie nadużywaj projektowania.
    \item Ale także nie pozostaw swojej pracy niezaprojektowanej.
    \item Unikaj błędów.
    \item Poczytaj o innych konwencjach, aby móc rozwinąć tę.
    \item Bądź konsekwentny.
    \item Używaj prostych rozwiązań.
  \end{enumerate}
\end{enumerate}

Stosowanie tej konwencji zapewni, że~napisany kod będzie zgodny z~ogólnym standardem.

\subsubsection{Nazewnictwo w~kodzie} \label{dokumentacja.nazewnictwo}

Aby zrozumieć istotę działania projektu należy zrozumieć mechanizmy i~algorytmy rządzące jego logiką. Zakładając, że~chcemy utworzyć dobrą dokumentację projektu, trzeba tak napisać kod, by~był zrozumiały -- jak pseudokod opisujący algorytm. W~tym celu wystarczy, że~wszelkie obiekty w~ kodzie, nazwiemy tak, by~ich użycie w~kontekscie było zrozumiałe a~ich rola jasna. W~trakcie pisania kodu najtrudniejszą kwestią jest nazywanie obiektów, a~nie -- jak to~się powszechnie uznaje -- rozwiązanie logiki systemu.

\subsection{Komentarze} \label{dokumentacja.komentarze}

Komentarze w~kodzie to~dobrzy sposób na~wyjaśnienie zasadności użycia algorytmów, bądź struktury API poszczególnych elementów kodu. Stosowanie komentarzy pozwala także wyróżnić, ważniejsze dla ,,czytelnika'' elementy kodu.

\subsubsection{RDoc} \label{dokumentacja.rdoc}

Istnieje wiele konwencji dotyczących~formy komentarzy. Format komentarzy ma~znaczenie nie tylko podczas czytania kodu -- może on~posłużyć pośrednio do~wygenerowania pełnej dokumentacji API przy użyciu odpowiednich narzędzi.


Komentarze w~języku Ruby mogą być dwojakiego formatu (listing \ref{code.comments}).

  \lstset{language=Ruby, caption=Komentarze w~języku Ruby, basicstyle=\ttfamily\footnotesize, numbers=left, numberstyle=\footnotesize, captionpos=b, backgroundcolor=\color{LightGray}}
  \begin{lstlisting}[label={code.comments}]
    # Komentarz jednolinijkowy rozpoczyna sie
    # od znaku krzyzyka.

    =begin
      Blok komentarza.
      Taki blok rozpoczyna sie od znacznika "=begin"
      a konczy na znaczniku "=end".
      Komentarze jednolinijkowe sa jednak czesciej
      uzywane.
    =end
  \end{lstlisting}


Narzędzie \textit{RDoc} \cite{rdoc} to~narzędzie konsolowe generujące dokumentację API w~zadanym formacie (standardowo jest to~HTML) na~podstawie kodu oraz~komentarzy w~nim zawartych. Aby wygenerować taką dokumentację wystarczy wpisać w~konsoli:

\mbox{\texttt{\$ rdoc <opcje> [plik...] }}

\textit{RDoc} wygeneruje czytelną dokumentację nawet bez komentarzy w~kodzie.

\subsection{Testy} \label{dokumentacja.testy}

Testy to~dodatkowe aplikacje sprawdzające poprawność logiki projektu. Zasadniczą ich funkcją jest dowiedzenie, że~dana funkcjonalność została zaimplementowana poprawnie.


Testować w~aplikacji można wszystko, jednak dobrze jest ustalić, co~chce się uzyskać poprzez napisanie testów. Biorąc pod uwagę powyższe kryterium testy dzieli się na:

\begin{enumerate}
 \item \textit{Testy jednostkowe} -- sprawdzają poprawność modułów ,,silnika'' aplikacji. Za~pomocą testów jednostkowych testuje się zwykle klasy, metody, stany maszyny, poprawność algorytmów, wejścia i~wyjścia strumieni.
 \item \textit{Testy behawioralne} -- odpowiadają na~pytanie: ,,co~się stanie, gdy zrobimy \ldots''. Testują zachowanie aplikacji -- reakcję na~żądania, efekty działania zdarzeń (takich jak, na przykład, wypełnienie formularza).
\end{enumerate}

\subsubsection{Testy jednostkowe RSpec} \label{dokumentacja.rspec}

\textit{RSpec} \cite{rspec} jest narzędziem do~tworzenia testów jednostkowych dla aplikacji napisanych w~języku Ruby. Jego składnia jest prosta, a~testy w~nim napisane -- czytelne (listing \ref{code.simple_test}).

  \lstset{language=Ruby, caption=Test \textit{RSpec} testujący klasę \texttt{Array}, basicstyle=\ttfamily\footnotesize, numbers=left, numberstyle=\footnotesize, captionpos=b, backgroundcolor=\color{LightGray}}
  \begin{lstlisting}[label={code.simple_test}]
  describe Array do
    describe "#push" do
      it "puts a value at the end of array" do
        array = Array.new
        value = "Some value"
        array.push value
        array.last.should == value
      end
    end
  end
  \end{lstlisting}

W~pierwszej kolejności podaje się opis testu -- krótką informację o~tym, co~testujemy i~jakie są oczekiwania względem testowanego obiektu. Wewnątrz bloku testu realizuje się przypadek użycia obiektu. W~każdym momencie można sprawdzić, czy interesujący nas stan obiektu jest zgodny z~naszymi oczekiwaniami. W~tym celu używa się metod \texttt{should} oraz~\texttt{should\_not}.


Testy uruchamiamy podając w~konsoli polecenie \texttt{rspec} oraz~plik z~napisanymi testami:

\mbox{\texttt{\$ rspec nazwa\_pliku\_testu.rb}}

Trzeba pamiętać, że~testy powinny mieć dostęp do~logiki projektu -- na~przykład~poprzez użycie instrukcji \texttt{require}. Efektem działania testu z~przykładu \ref{code.simple_test} widoczny jest na listingu \ref{code.simple_test_results}

\lstset{language=bash, caption=Efekt uruchomienia testu jednostkowego z~przykładu \ref{code.simple_test}, basicstyle=\ttfamily\footnotesize, numbers=left, numberstyle=\footnotesize, captionpos=b, backgroundcolor=\color{LightGray}}
\begin{lstlisting}[label={code.simple_test_results}]
.

Finished in 0.00258 seconds
1 example, 0 failures
\end{lstlisting}

Aby zrozumieć, co~stało się podczas testowania, można zmienić format wyjścia na~bardziej przyjazny (listing \ref{code.test_pretty}).

\lstset{language=bash, caption=Efekt uruchomienia testu jednostkowego z~opcją \texttt{----format d}, basicstyle=\ttfamily\footnotesize, numbers=left, numberstyle=\footnotesize, captionpos=b, backgroundcolor=\color{LightGray}}
\begin{lstlisting}[label={code.test_pretty}]
Array
  #push
    puts a value at the end of array

Finished in 0.0029 seconds
1 example, 0 failures
\end{lstlisting}

Jak widać \texttt{rspec} poinformował o~tym, które testy ,,przeszły'' -- czyli, które z~testowanych funkcjonalności spełniły oczekiwania.

\subsubsection{RSpec jako dokumentacja}

Przykład \ref{code.simple_test} mówi dość dużo o~sposobie użycia elementów logiki projektu. Testy jednostkowe są~tak naprawdę przypadkami użycia tychże elementów. Co~więcej -- przypadki te~realizują się zgodnie z~założeniami twórcy takiego projektu (uruchomione testy ,,przechodzą''). Oznacza to, że~są one dobrą dokumentacją działania i~sposobu użycia poszczególnych obiektów logiki projektu.

\subsubsection{Dokumentacja poprzez testy behawioralne} \label{dokumentacja.cucumber}

Skoro testy jednostkowe są~swego rodzaju dokumentacją, to~także testy behawioralne mogą nią być. Oczywiście ze~względu na~inny cel testy behawioralne będą dokumentować projekt z~innej perspektywy. Testy te~określają zachowanie aplikacji, a~zatem kwestię ściśle związane z~użytkowaniem. W~praktyce testy behawioralne są~stosowane dla określenia wymagań ze~strony klienta -- zleceniodawcy projektu.


Istnieje kilka narzędzi, przy pomocy których można napisać i~przeprowadzić testy behawioralne. Można do~tego użyć \texttt{rspec} oraz~\texttt{Test::Unit} (czyli narzędzi wykorzystywanych przy pisaniu testów jednostkowych). Dla dobrej czytelności oraz~eleganckiego formatu użyto narzędzia \textit{Cucumber} \cite{cucumber}.


\section{Podstawowe definicje}
Ten podrozdział powinien zawierać dokładny opis terminologii  pojęć zasadniczych dla tematu pracy, którymi autor będzie się posługiwał przy realizacji głównych celów pracy. 


\section{Istniejące rozwiązania w dziedzinie}
W tym podrozdziale zostaną opisane.....
\subsection{Sprzęt}
.........................
\subsection{Oprogramowanie i wdrożone systemy}
.....................................
\subsection{}
...................

\section{Wady i słabe punkty istniejących rozwiązań}
\subsection{Efektywność}
..........................
\subsection{Utrudniony dostęp}
..............
\subsection{Wysokie koszty}
...............................

\chapter{Dalsze uwagi o edycji i~formatowaniu pracy}
Pracę w \LaTeX'u najlepiej składać w szablonie {\tt report}, ze względu na jendostronny wydruk (jak w {\tt article}) i możliwość dzielenia pracy na rozdziały, a co za tym idzie, tworzenia spisu treści, spisu tabel, rysunków. 
\begin{example}
Przyklad
\end{example}

\begin{corollary}
Wniosek
\end{corollary}
\section{Bibliografia i przypisy}
Spis litertury dołącza się  w \LaTeX'u automatycznie na końcu pracy (zob.
komenda {\tt begin{thebibliography}}). Informacje o sposobie cytowania zawarte
są na stronie Bibilioteki Głównej PŁ\\ także udostępnione na
\underline{\tt http://ics.p.lodz.pl/\textasciitilde aniewiadomski}. \\

\indent Przykład cytowania --  jak podaje praca \cite{kacprzyk86}, ......,
jednakże autorzy [2] twierdzą, iż.....\\


\indent Za każdym razem, kiedy w pracy pojawia się treść na podstawie jakiegoś
tekstu źródłowego czyjegoś autorstwa, oznaczamy takie miejsce
przypisem\footnote{Treść przypisu pierwszego}. Przypis zawierać musi  numer
jakim w spisie literatury, czyli bibliografii, oznaczono tę pracę, np.
tak\footnote{[3], ss. 3--6 (czyli praca trzecia w spisie literatury,
wykorzystany fragment znajduje się na stronach od 3. do 6.)}. {\bf
Wszystkie źródła tekstów, rysunków, danych, wykresów, schematów, kodów i
informacji wykorzystanych w pracy muszą być zamieszczone w bibliografii.
Wszystkie pozycje literatury zamieszczone w bibliografii muszą być cytowane w
treści pracy, na dowód, iż zostały rzeczywiście użyte przy pisaniu pracy.}

\subsubsection{Źródła elektroniczne}
Źródła elektroniczne, zwłaszcza internetowe należy cytować z należytą uwagą na
ich jakość. Nie cytujemy źródeł wątpliwej jakości lub wtórnie przekazujących
czy też powielających wiedzę zawartą w innych źródłach, np. fora internetowe lub
wikipedia.\\
\indent {\bf Wszystkie wykorzystane źródła elektroniczne powinny być przez Autora
pracy skopiowane \underline{w dniu ich wykorzystania} i
dołączone np. na CD/DVD do wersji drukowanej pracy.\\
\indent Odnośniki do źródeł elektronicznych muszą zawierać pełną ścieżkę, np. do
pliku lub rysunku, a nie jedynie domenowy adres portalu, np.\newline {\tt
http://serwer.com/temat/podtemat/katalog/plik\_strony.html} (stan na dzień:
2009-12-05)\\
ale nie\\
{\tt www.portal.pl}.} (!!!!!)

Niedochowanie tego wymogu może stać się powodem odrzucenia pracy ze wzglę\-dów
formalnych (,,brak możliwości weryfikacji źródeł wykorzystanych w pracy''). 

\section{Polskie akapity, cudzysłowy, itp.}
Akapity stosujemy zawsze z wcięciem, ale bez wiersza odstępu pomiędzy akapitami. 
Ta forma jest przyjęta dla publikacji polskojęzycznych. {\bf W szczególnych
przypadkach (także w tym szablonie)} akapit występujący bezpośrednio po tytule
rozdziału, sekcji, podsekcji itp. NIE JEST WCIĘTY.\\
\indent Ten akapit JEST WCIĘTY. NIE MA także PUSTEGO WIERSZA pomiędzy tym
akapitem a poprzednim. \\
\indent Podobne uwagi dotyczą wszystkich innych elementów formatowania pracy --
muszą być zgodne ze zwyczajami przyjętymi W JĘZYKU POLSKIM.	Np. cu\-dzysłowy
wyglądają tak: ,,cudzysłów'', ale nie ''cudzysłów'', albo też `cudzysłów' czy
``cudzysłów''. 



\section{Definicje i wyrażenia matematyczne}
\begin{definition} \label{def.definicja1}
Niech $\cal X$ będzie przestrzenią.....
\end{definition}

Do definicji odnieść sie można poprzez jej etykietę: jak podano w Def.~\ref{def.definicja1}

Przykładowe podkreślenie... \underline{tekst podkreślony}, pogrubienie: {\bf
tekst pogrubiony} oraz wyrożnienie {\em tekst wyróżniony, czyli kursywa}. 
Dalszy tekst rozdziału
Dalszy tekst rozdziału
Dalszy tekst rozdziału
Dalszy tekst rozdziału
Dalszy tekst rozdziału
Dalszy tekst rozdziału
Dalszy tekst rozdziału
Dalszy tekst rozdziału
Dalszy tekst rozdziału a teraz koniec linii... \\
\indent ... i nowy akapit. Akapity muszą być standardowo wcięte.  


Przykład wzoru matematycznego numerowanego

\begin{equation} \label{wzoreinsteina}
E=m\cdot c^2
\end{equation}

{\bf Wszystkie symbole matematyczne występujące w tekście ,,na bieżąco'',
czyli nieoznaczone numerem równania TAKŻE PISZEMY W TRYBIE MATEMA\-TYCZNYM, CZYLI
K U R S Y W Ą} : $a=b\cdot c$, ale nie: a= b*c (!!)\\

\indent Numeracja wzoru -- ZAWSZE w POSTACI (\#.\#\#)
Jak podaje wzor (\ref{wzoreinsteina}).... (koniec linii). \\
\indent Wyrażenia matematyczne można też wpisywać w wierszu -- używamy wów\-czas znaku '\$', który rozpoczyna i kończy wyrażenie, np. wg Einsteina $E=m\cdot c^2$...


\section{Jak wstawiać rysunki? tabele? }
A teraz pora na rysunek:
\begin{figure}[!t]
\centering

\caption{Funkcja przynależności zbioru rozmytego -- Podpis ZAWSZE POD rysunkiem,
numeracja w postaci \#.\#\#. } (wypada podać źródło, czyli literaturę,
z której rysunek pochodzi, ewentualnie {\em opracowanie własne}.)
\label{fig.funkcja.przyn}
\end{figure}


\begin{table}[!t]
\centering
\caption{Tytuł tabeli ZAWSZE NAD TABELĄ, numeracja w formie \#.\#\#. (wypada podać źródło, czyli literaturę,
z której tabela pochodzi, ewentualnie {\em opracowanie własne}.)} 

\label{tabls1}

{\footnotesize 
\vspace{5mm}
\begin{tabular}{c c c c c}
\hline\noalign{\smallskip}
{\bf Alg.} & {\bf tytuł kolumny 1} & {\bf tytuł kolumny 1} & {\bf Tytuł kolumny
3} & {\bf ....}     \\

\hline\noalign{\smallskip}
a & b & c & d & e  \vspace{3mm} \\ 
\noalign{\smallskip}
 a & b & c & d & e \\

\noalign{\smallskip}
%%%
\end{tabular}
}
\end{table}


Rysunki i tabele nie powinny przekraczać 0.9 szerokości tekstu i zasadniczo
powinny występować na górze strony. \\

\indent Odnosić się do rysunku można poprzez jego etykietę ''label'', np. jak widać na rys. \ref{fig.funkcja.przyn}......

Jak widać, rysunek nie wypada w dokumencie w tym samym miejscu co w kodzie, choć czasem się tak zdarza. Jeśli potrzebujesz przenieść rysunek, zajrzyj do rozdzialu 2.11. manuala pt. {\em Wstawki}. 

\section{Listy wypunktowana i numerowana}

\begin{itemize}
\item pierwszy element listy wypunktowanej
\item drugi...
\item trzeci...
\end{itemize}


Nowy akapit z lista numerowaną. 
\begin{enumerate}
 \item pierwszy element listy NUMEROWANEJ
 \item drugi...
 \item trzeci...
 \item trzeci...
 \item trzeci...
 \end{enumerate}

\section{Przenoszenie wyrazów}
Skorzystaj z polecenia {\tt hyphenation}\\ w preambule dokumentu, lub dziel
wyrazy ,,ręcznie'' czyli właśnie tak jak tu: po\-dzie\-lo\-ne wy\-ra\-zy. 

\chapter[Technologie i metody użyte...]{Technologie i metody użyte w~części
badawczej}

{\em Tytuł tego rozdziału ma dwie wersje: zwykłą, (w kodzie: w nawiasach
klamrowych), która
pokazuje sie na stronie rozpoczynającej rozdział, oraz krótką (w kodzie: w nawiasach
kwadratowych), która pokazuje sie w spisie treści i w nagłówku}

W rozdziale \ref{etykietarozdzialu2} podano podstawy teoretyczne i ogólny zakres
pracy. W niniejszym rozdziale opisana zostanie technologia XYZ oraz metoda ABC
użyta w części praktycznej, patrz rozdział~\ref{rozdz.czesc.prakt}. 

\section{Sprzęt}
...................
\subsection{Element 1}
.........................
\subsection{Element 2}
......................

\section{Oprogramowanie}
..........................
\subsection{Serwer baz danych}
........................
\subsection{Środowisko zintegrowane}
..........................
\subsection{Oprogramowanie klienckie}

\section{Technologie i metodologie programistyczne}
..................
\subsection{Język programowania}
......................
\subsection{Biblioteki}
.......................
\subsection{Wzorce projektowe}
.......................

\section{Inne, np. narzędzia i metody symulacji, }

\chapter{Aplikacja/system/projekt "XYZ"} \label{rozdz.czesc.prakt}
Ta część pracy może być podzielona na więcej rozdziałów, np kiedy autor chce
w~szczególności podkreślić któryś z etapów projektu. W zależności od tematu i~celów pracy, pewne sekcje można dodać (np. przy projektowaniu sieci, instalacji
i~konfiguracji serwerów usług sieciowych), inne zaś pominąć.

\section{Analiza wymagań}
\subsection{Studium możliwości}
\subsection{Wymagania funkcjonalne}
.................
\subsection{Ograniczenia projektu}

\section{Projekt}
\subsection{Projekt warstwy danych}

\begin{enumerate}
\item normalizacje baz danych
\item projekt bazy/baz 
\item grupy użytkowników i ich prawa dostępu do danych (zależne od implementacji bazy)
\item ew. diagramy klas warstwy danych
\end{enumerate}
\subsection{Projekt warstwy logiki}
\begin{enumerate}
\item Diagramy i scenariusze przypadków użycia
\item Diagramy przepływu danych (lub ich odpowiedniki)
\item ew. diagramy klas, wzorce projektowe itp.
\end{enumerate}

\subsection{Projekt warstwy interfejsu użytkownika}
\subsubsection{Wybór środowiska i platformy działania}
\subsubsection{Rodzaj aplikacji (klient-serwer, thick/thin client, aplikacja
,,biurkowa'', usługa, klient hybrydowy, itp.}
\subsubsection{Technologie projektowania i realizacji interfejsu użytkownika,
np. biblioteki}


\section{Implementacja: punkty kluczowe}

\section{Testy i wdrożenie}
\subsection{Testy wydajności}
\subsection{Testy regresyjne}
\subsection{Testy bezpieczeństwa}
\subsection{Dalsze testy}
\subsection{Testy...}

\section{Konserwacja i inżynieria wtórna}
Jak przebiega eksploatacja systemu/projektu? Jakie wady i zalety ujawniły się po
np. 2-miesięcznym okresie testowania i użytkowania? \\
\indent Jak można skorzystać z tej wiedzy praktycznej pod kątem roz\-bu\-do\-wy pracy? Jakie elementy systemu powinny zostać w pierwszej kolejności zmodyfikowane?  

\chapter{Podsumowanie}
\section{Dyskusja wyników}
Dzięki zrealizowaniu pracy poprawie uległa wydajność ....... Ponadto, o ?? \%
skrócony został czas ........, a koszty osiągnięcia zamierzonego efektu zostały
zmniejszone z ???pln do ???pln za godzinę/ dzień/ jednostkę sprzętu.........\\
\indent Które cele pracy udało sie zrealizować? co z tego wynika? Które cele
pracy pozostały niezrealizowane i dlaczego? 

\section[Ocena możliwości wdrożenia...]{Ocena możliwości wdrożenia proponowanych
\newline rozwiązań...}
... ich wartość praktyczna, lokalne i globalne możliwości zastosowania, kwestia
praw autorskich do powstałych produktów, itp. 

\section{Perspektywy dalszych badań w dziedzinie}
Jak można kontynuować tę pracę, zwłaszcza pod kątem studiów
uzupełniających magisterskich i/lub doktoranckich. Co jeszcze powinno być
zrobione lub ulepszone? Co należy zmienić lub poprawić w pracy z dzisiejszego punktu widzenia?


\addcontentsline{toc}{chapter}{Bibliografia} 
\begin{thebibliography}{99}
\bibitem{kacprzyk86}
Kacprzyk J. (1986) Fuzzy sets in system analysis.  PWN, Warsaw (in Polish).
\bibitem{kacprzyk99b}
Kacprzyk J., Strykowski P. (1999) Linguistic Data Summaries for Intelligent Decision Support, Proceedings of EFDAN'99. 4-th European Workshop on Fuzzy Decision Analysis and Recognition Technology for Management, Planning and Optimization, Dortmund, 1999, 3--12.
\bibitem{kacprzyk01d}
Kacprzyk J., Yager R. R. (2001) Linguistic summaries of data using fuzzy logic. International Journal of General Systems 30:133--154 

\end{thebibliography}

\addcontentsline{toc}{chapter}{Spis rysunków} 
\listoffigures

\addcontentsline{toc}{chapter}{Spis tabel} 
\listoftables


\addcontentsline{toc}{chapter}{Załączniki} 
\chapter*{Załączniki}
\begin{enumerate}
\item Załącznik nr 1
\item Załącznik nr 2
\item Załącznik nr 3
\end{enumerate}


\end{document}
