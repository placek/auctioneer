\section{Dokumentacja projektu} \label{dokumentacja}

Projekty systemów informatycznych mają to~do~siebie, że~rozwijają się niezwykle szybko. Rozwój projektu związany jest niestety z~powiększaniem objętości projektu -- zarówno merytorycznej, jak i~czysto fizycznej (np.~ilość linijek kodu, ilość plików, itp.). Problem pojawia się, gdy projekt jest zbyt ,,duży'' żeby programista mógł rozumieć wszystko to, co~się w~nim dzieje. Rozwiązaniem dla tego problemu jest tworzenie dokumentacji.


W~projektach typu OpenSource dokumentacja jest niezwykle ważnym czynnikiem usprawniającym pracę programistów w~nich pracujących. Dobrze napisana dokumentacja sprawia, że~nowe osoby zaczynające pracę w~projekcie mogą łatwiej i~szybciej zapoznać się ze~strukturą, działaniem oraz~organizacją projektu. Jest to~szczególnie przydatne, gdy istnieje potrzeba edycji jedynie niewielkiego fragmentu projektu. Co~więcej -- osoby zajmujące się już od~dłuższego czasu pracą w~takim projekcie nie muszą pamiętać wszelkich zagadnień z~nim związanych. Pozwala to~programiście na~pracę w~wielu projektach na~raz.


\subsection{Kod aplikacji} \label{dokumentacja.kod}

Najlepszą dokumentacją dla systemu informatycznego jest kod aplikacji. To~właśnie on~realizuje wszystkie założenia projektowe, algorytmy czy cykle pracy naszego projektu. Prawidłowo napisany kod może powiedzieć więcej niż nie jedna dokumentacja -- tu~dowiadujemy się jak naprawdę działa interesujący nas moduł i~mamy pewność, że~nie padniemy ofiarą nieporozumień. Pod pojęciem ,,prawidłowo napisany kod'' mam na~myśli spełnienie następujących założeń:

\begin{enumerate}
  \item trzymanie się konwencji określającej osnowę dokumentu zawierającego kod (np.~konsekwencję w stosowaniu wcięć),
  \item stosowanie zrozumiałych nazw dla wszelkich struktur (nazwy klas, metod, zmiennych, itp.),
  \item tworzenie krótkich i~treściwych fragmentów kodu oraz~rozbijanie większych partii na~mniejsze.
\end{enumerate}

Wszystkie te kroki mają na celu sprawienie, że~kod stanie się bardziej czytelny. Omówię teraz te~trzy kroki nieco dokładniej w zastosowaniu dla języka Ruby.


\subsubsection{Konwencja} \label{dokumentacja.konwencja}

Język Ruby ze~względu na~swoją składnię jest niezwykle czytelny, a~co~za~tym idzie, ułatwia dokumentację projektu w~nim napisanego. Składnia Rubiego przypomina pseudokod:

  \lstset{language=Ruby, caption=Przykład prostej składni języka Ruby -- algorytm DFS., basicstyle=\ttfamily\footnotesize, numbers=left, numberstyle=\footnotesize, captionpos=b, backgroundcolor=\color{LightGray}} \label{code.simpleruby}
  \begin{lstlisting}
  def dfs node, value, queue
    return false if node.nil?
    return true if node.data == value
    queue.push node.right_neighbor unless node.right_neighbor.nil?
    queue.push node.left_neighbor unless node.left_neighbor.nil?
    dfs queue.pop, value, queue
  end
  \end{lstlisting}

Niestety -- sama składnia nie jest najważniejsza w~dokumentowaniu projektów. Powyższy przykład można przecież przepisać w~następujący sposób:

  \lstset{language=Ruby, caption=Algorytm DFS z~zastosowaniem braku konwencji formatu., basicstyle=\ttfamily\footnotesize, numbers=left, numberstyle=\footnotesize, captionpos=b, backgroundcolor=\color{LightGray}}
  \begin{lstlisting}
  def dfs node, value, queue
  return false if node.nil?; if node.data == value
  return true
  end; queue.push node.right_neighbor unless \\
    node.right_neighbor.nil?
  unless node.left_neighbor.nil?
  queue.push node.left_neighbor; end
  dfs queue.pop, value, queue
  end
  \end{lstlisting}

Jak widać przykład ten jest mniej czytelny, a~co~za~tym idzie, trudniej jest dowiedzieć się, za~co~dany fragment kodu jest odpowiedzialny. W~celu uzyskania ,,przejrzystości'' kodu stosuje się konwencje zapisu -- ogólnie przyjęte zasady mówiące o~tym jak powinien wyglądać kod i~jak ten kod formatować. Jest to~swego rodzaju etykieta -- zbiór zasad obowiązujących dla języka. Każdy język ma~swoją konwencję (a~czasem nawet kilka). Język Ruby także ,,dorobił się'' swojej. Konwencja ta~nosi nazwę \textit{The Ruby Style}\footnote{Patrz: \url{https://github.com/chneukirchen/styleguide/blob/master/RUBY-STYLE}} i~określona jest następująco (tłumaczenie własne):

\begin{enumerate}
  \item Formatowanie:
  \begin{enumerate}
    \item Używaj zawsze kodowania ASCII lub UTF8.
    \item Używaj dwóch spacji jako wcięć (nigdy tabulatur).
    \item Staraj się kończyć wiersz w~stylu używanym w~systemach Unix (LF -- 0x0A).
    \item Używaj spacji przed i~po~operatorach, po~przecinkach, po~dwukropkach, po~średnikach, przed i~po~\{ oraz~po~\}.
    \item Nie używaj spacji po~( oraz~[. Nie używaj spacji przed~] oraz~).
    \item Używaj dwóch spacji przed modyfikatorem warunku (\texttt{if}/\texttt{unless}/\texttt{while}/\texttt{until}/\texttt{resque}).
    \item Wcięcie dla słowa kluczowego \texttt{when} powinno być tak głębokie jak dla \texttt{case}.
    \item Użyj pustego wiersza przed zwracaną wartością w~metodzie (chyba, że~ma~ona tylko jeden wiersz), a~także przed słowem kluczowym \texttt{def}
    \item Używaj RDoc do~dokumentacji API. Nie wstawiaj pustego wiersza pomiędzy komentarz a~komentowany blok.
    \item Użyj pustego wiersza do~podzielenia dużych metod na~logiczne fragmenty.
    \item Staraj się by~każdy wiersz miał mniej niż 80 znaków.
    \item Unikaj białych znaków na~końcu wiersza.
  \end{enumerate}
  \item Składnia:
  \begin{enumerate}
    \item Użyj \texttt{def} z~ nawiasami, gdy są~podane argumenty.
    \item Nie używaj \texttt{for}, chyba, że~robisz to~celowo.
    \item Nie używaj \texttt{then}.
    \item Użyj ,,\texttt{when x;} \ldots'' dla jednolinijkowego wyrażenia \texttt{case}.
    \item Użyj \texttt{\&\&}/\texttt{||} dla wyrażeń boolowskich, \texttt{and}/\texttt{or} do~kontroli przepływu. (Ogólna zasada: jeśli używasz nawiasów, to~znaczy, że~używasz złych operatorów).
    \item Unikaj wielolinikowej instrukcji \texttt{?:}, użyj \texttt{if}.
    \item Zaniechaj użycia nawiasów przy wywołaniu metod, ale użyj ich podczas wywołania ,,funkcji'' (np.~gdy używasz zwracanej wartości w~tym samym wierszu)
    \item Używaj raczej \mbox{\texttt{\{} \ldots \texttt{\}}} niż \mbox{\texttt{do} \ldots \texttt{end}}. Wielolinijkowe bloki \mbox{\texttt{\{} \ldots \texttt{\}}} są~w~porządku: używając \texttt{\}} na końcu bloku wiemy, że~kończy się blok a~nie instrukcja \texttt{if}/\texttt{while}/\ldots. Używaj \mbox{\texttt{do} \ldots \texttt{end}} do~kontroli przepływu (np.~zadania \texttt{rake}, bloki \texttt{sinatra}).
    \item Unikaj używania słowa kluczowego \texttt{return} jeśli nie jest potrzebne.
    \item Unikaj kontynuacji linii (\texttt{$ \backslash $}) jeśli nie musisz.
    \item Używanie zwracanej wartości przez operator \texttt{=} jest na miejscu.
    \item Używaj operatora \texttt{||=}.
    \item Używaj wyrażeń regularnych typu ,,non-OO''.  Nie bój się używać \texttt{=~}, \texttt{\$0-9}, \texttt{\$~}, \texttt{\$`} oraz~\texttt{\$'} jeśli potrzebujesz.
  \end{enumerate}
  \item Nazewnictwo:
  \begin{enumerate}
    \item Używaj snake\_case jako stylu nazywania metod.
    \item Używaj CamelCase jako stylu nazywania klas i~modułów (Pozostaw akronimy takie jak HTTP, RFC, XML z~wielkich liter).
    \item Używaj SCREAMING\_SNAKE\_CASE jako stylu nazywania stałych.
    \item Długość nazwy zwykle określa kontekst wykorzystania. Używaj jednoliterowych zmiennych jako parametrów bloków/metod według tego schematu:\\
      \texttt{a}, \texttt{b}, \texttt{c}, \texttt{o}: dowolny obiekt;\\
      \texttt{d}: katalog;\\
      \texttt{e}: element (klasy Emumerable oraz~pochodnych);\\
      \texttt{ex}: wyjątek;\\
      \texttt{f}: plik:\\
      \texttt{i}, \texttt{j}: indeksy;\\
      \texttt{k}: klucz tablicy asocjacyjnej;\\
      \texttt{m}: metoda;\\
      \texttt{r}: zwracana wartość krótkich metod;\\
      \texttt{s}: tekst;\\
      \texttt{v}: wartość elementu tablicy asocjacyjnej;\\
      \texttt{x}, \texttt{y}, \texttt{z}: liczby;\\
      Ponadto pierwsza litera klasy obiektu może posłużyć za~nazwę takiej zmiennej.

    \item Używaj nazw zaczynających się od~\texttt{\_} dla nieużywanych zmiennych.
    \item Używając \texttt{inject} dla krótkich bloków nazywaj argumenty \texttt{|a, e|} (od:~akumulator, element).
    \item Definiując operatory dwuargumentowe, nazywaj argument jako ,,other''.
    \item Używaj raczej \texttt{map} niż \texttt{collect}, \texttt{find} niż \texttt{detect}, \texttt{find\_all} niż \texttt{select} oraz~\texttt{size} niż \texttt{length}.
  \end{enumerate}
  \item Komentarze:
  \begin{enumerate}
    \item Komentarze dłuższe niż słowo rozpoczynają się z~wielkiej litery i~używane są zasady interpunkcji.
    \item Używaj dwóch spacji po każdej kropce w komentarzu.
    \item unikaj nie potrzebnych komentarzy.
  \end{enumerate}
  \item Pozostałe:
  \begin{enumerate}
    \item Pisz kod przyjazny dla opcji \texttt{ruby -w}.
    \item Unikaj tablic asocjacyjnych jako opcjonalnych parametrów. Być może metoda, którą piszesz robi zbyt wiele?
    \item Unikaj długich metod.
    \item Unikaj długich list parametrów.
    \item Używaj konstrukcji \texttt{def self.}\textit{metoda} dla zdefiniowania metod singletonów.
    \item Staraj się rozwijać funkcjonalność standardowych metod.
    \item Unikaj \texttt{alias} -- używaj \texttt{alias\_method}.
    \item Używaj \texttt{OptionParser} do~parsowania skomplikowanych opcji wejścia konsoli a~\texttt{ruby -s} dla rozwiązań trywialnych.
    \item Staraj się zachować zgodność wielu wersji interpretera.
    \item Unikaj zbędnego metaprogramowania.
  \end{enumerate}
  \item Ogólne zasady:
  \begin{enumerate}
    \item Programuj w sposób funkcjonalny.
    \item Nie wydziwiaj z~używaniem argumentów metod -- chyba, że~wiesz co robisz.
    \item Nie zmieniaj funkcjonalności standardowych bibliotek pisząc własne.
    \item Nie programuj zachowawczo\footnote{Patrz: \url{http://www.erlang.se/doc/programming\_rules.shtml\#HDR11}}.
    \item Postaraj się zachować prostotę kodu.
    \item Nie przesadź z~projektowaniem.
    \item Ale także nie pozostaw swojej pracy niezaprojektowanej.
    \item Unikaj błędów.
    \item Poczytaj o innych konwencjach, aby móc rozwinąć tę.
    \item Bądź konsekwentny.
    \item Używaj prostych rozwiązań.
  \end{enumerate}
\end{enumerate}

Stosowanie tej konwencji zapewni, że~kod napisany przez nas będzie zgodny z~ogólnym standardem, którego używają Rubyści na~całym świecie.

\subsubsection{Nazewnictwo w~kodzie} \label{dokumentacja.nazewnictwo}

Aby zrozumieć istotę działania projektu należy zrozumieć mechanizmy i~algorytmy rządzące jego logiką. Zakładając, że~chcemy dobrze odokumentować nasz projekt musimy tak napisać kod, by~był zrozumiały -- jak pseudokod opisujący algorytm. W~tym celu wystarczy, że~wszelkie obiekty, jakie tylko używamy w~naszym kodzie, nazwiemy tak, by~ich użycie w~kontekscie było zrozumiałe a~ich rola nie pozostawiała miejsca na zastanowienie. W~trakcie pisania kodu najtrudniejszą kwestią jest nazywanie obiektów a~nie -- jak to~się powszechnie uznaje -- rozwiązanie logiki systemu. Szczęśliwie dla programistów języka Ruby można stosować nawet bardzo długie nazwy obiektów (oczywiście nie wolno przesadzać) pozwalające opisać zastosowanie danego obiektu w~kodzie.


Najlepszy do wyjaśnienia tej sytuacji będzie przykład:

  \lstset{language=Ruby, caption=Algorytm DFS z~nazwami zmiennych o~niejasnym znaczeniu., basicstyle=\ttfamily\footnotesize, numbers=left, numberstyle=\footnotesize, captionpos=b, backgroundcolor=\color{LightGray}}
  \begin{lstlisting}
  def wwd a, w, ko
    return false if a.nil?
    return true if a.d == v
    ko.push a.r unless a.r.nil?
    ko.push a.l unless a.l.nil?
    dfs ko.pop, w, ko
  end
  \end{lstlisting}

Taki kod nie jest czytelny. Potencjalny ,,czytelnik'' musi przemyśleć działanie kodu a~i~tak nie ma~pewności czy zrozumie przeznaczenie kodu.

  \lstset{language=Ruby, caption=Algorytm DFS z~nazwami zmiennych nie mówiących o~przeznaczeniu kodu., basicstyle=\ttfamily\footnotesize, numbers=left, numberstyle=\footnotesize, captionpos=b, backgroundcolor=\color{LightGray}}
  \begin{lstlisting}
  def dfs element, value, list
    return false if element.nil?
    return true if element.data == value
    list.push element.right unless element.right.nil?
    list.push element.left unless element.left.nil?
    dfs list.pop, value, list
  end
  \end{lstlisting}

Tu kod został opatrzony lepszymi nazwami. Co~prawda ,,czytelnik'' jest w~stanie domyśleć się (i~to~dość szybko) co~ten kod ,,robi'', ale przeznaczenie kodu nadal nie jest znane.

Nasz pierwszy przykład stanowi kod napisany w~sposób przyjazny potencajnemu ,,czytelnikowi''. W~tym przykładzie widać już ,,na~pierwszy rzut oka'' jaką rolę spełnia kod, jakie jest jego przeznaczenie, a~w razie potrzeby można taki kod rozbudować (np.~w~celu dodania funkcjonalności).


Aby całkowicie opisać problem nazewnictwa obiektów w~kodzie przedstawiam ostatni przykład:
  \lstset{language=Ruby, caption=Algorytm DFS -- mylące nazwy zmiennych., basicstyle=\ttfamily\footnotesize, numbers=left, numberstyle=\footnotesize, captionpos=b, backgroundcolor=\color{LightGray}}
  \begin{lstlisting}
  def st element, value, stack
    return false if element.nil?
    return true if element.value == value
    stack.push element.predecessor unless element.predecessor.nil?
    stack.push element.follower unless element.follower.nil?
    dfs stack.pop, value, stack
  end
  \end{lstlisting}

W~tym przykładzie natomiast nazwy zostały specjalnie zmienione tak by~ukryć przeznaczenie przed oczyma potencjalnego ,,czytelnika''. W poprzednim przykładzie widzieliśmy jaki jest cel tego fragmentu kodu. Tutaj jesteśmy zakłopotani, gdyż nie rozumiemy przeznaczenia tego algorytmu.

\subsubsection{Zasada dziel i~zdobywaj} \label{dokumentacja.dzielizdobywaj}

W~\textit{The Ruby Style} czytamy, że~powinniśmy unikać długich metod. Dlaczego długie metody są~złe? Ponieważ utrudniają czytanie i~zniechęcają do~zapoznania się z ich treścią. Podam tutaj przykład takiej metody:

  \lstset{language=Ruby, caption=Obszerna metoda wykonująca algorytm Levenshtein'a., basicstyle=\ttfamily\footnotesize, numbers=left, numberstyle=\footnotesize, captionpos=b, backgroundcolor=\color{LightGray}}
  \begin{lstlisting}
def levenshtein s, t
  cost = 0
  d = Array.new
  m = s.length
  n = t.length
  0.upto(m) do |i|
    d[i] = Array.new
  end
  0.upto(m) do |i|
    d[i][0] = i
  end
  1.upto(n) do |j|
    d[0][j] = j
  end
  1.upto(m) do |i|
    1.upto(n) do |j|
      cost = (s[i-1,1] == t[j-1,1]) ? 0 : 1
      d[i][j] = [(d[i - 1][j] + 1), (d[i][j - 1] + 1), \\
        (d[i - 1][j - 1] + cost)].min
    end
  end
  return d[m][n]
end
  \end{lstlisting}

Metoda ta~może zostać zapisana w~prostszy sposób. Wystarczy logiczne fragmenty metody zamienić na~inne metody:

  \lstset{language=Ruby, caption=Metoda z~poprzedniego przykładu podzielona na~mniejsze metody., basicstyle=\ttfamily\footnotesize, numbers=left, numberstyle=\footnotesize, captionpos=b, backgroundcolor=\color{LightGray}}
  \begin{lstlisting}
def levenshtein s, t
  a = initiate s.length, t.length
  calculate a
end

def initiate m, n
  array = Array.new
  0.upto(m) do |i|
    array[i] = [i]
  end
  1.upto(n) do |j|
    array.first[j] = j
  end
  array
end

def calculate array
  cost = 0
  1.upto(array.size) do |i|
    1.upto(array.first.size) do |j|
      cost = (s[i - 1][1] == t[j - 1][1]) ? 0 : 1
      d[i][j] = [(d[i - 1][j] + 1), (d[i][j - 1] + 1), \\
        (d[i - 1][j - 1] + cost)].min
    end
  end
  array.last.last
end
  \end{lstlisting}

Co przy tym zyskujemy? Metoda nas interesująca jest krótsza -- łatwiej się ją~czyta, łatwiej znaleźć w~niej potencjalne błędy. Można także opisać kod stosując odpowiednie nazwy metod wywoływanych przez naszą metodę -- dzięki temu nie musimy zagłębiać się dokładnie w~działanie metody ale poznajemy kolejność wykonywanych w~niej operacji. Podział taki może nawet paradoksalnie zmniejszyć objętość naszego kodu -- jedna z~metod może wykonać się kilkukrotnie w~danej części kodu. W~przyszłości może to~pomóc nawet w~optymalizacji kodu.

\subsubsection{Zen programisty języka Ruby} \label{dokumentacja.zenruby}

,,Właściwie napisany kod staje się dobrą dokumentacją samego siebie''. Programiści języka Python byli jednymi z~pierwszych i~najbardziej zagorzałych zwolenników tej koncepcji. Aby to zaakcentować powstał krótki dokument o nazwie ,,The Zen of Python'' autorstwa Tima Petersa \cite{zenpython}. Dokument ten można przeczytać wpisując w~konsoli języka Python polecenie: \texttt{import this}. Oto treść tego dokumentu \footnote{Tłumaczenie wzięte z \url{http://pl.python.org/forum/index.php?topic=392.0}}:

\begin{enumerate}
  \item Piękne jest lepsze niż brzydkie.
  \item Wyrażone wprost jest lepsze niż domniemane.
  \item Proste jest lepsze niż złożone.
  \item Złożone jest lepsze niż skomplikowane.
  \item Płaskie jest lepsze niż wielopoziomowe.
  \item Rzadkie jest lepsze niż gęste.
  \item Czytelność się liczy.
  \item Sytuacje wyjątkowe nie są~na~tyle wyjątkowe, aby łamać reguły.
  \item Choć praktyczność przeważa nad konsekwencją.
  \item Błędy zawsze powinny być sygnalizowane.
  \item Chyba że zostaną celowo ukryte.
  \item W razie niejasności powstrzymaj pokusę zgadywania.
  \item Powinien być jeden -- i~najlepiej tylko jeden -- oczywisty sposób na~zrobienie danej rzeczy.
  \item Choć ten sposób może nie być oczywisty jeśli nie jest się Holendrem.
  \item Teraz jest lepsze niż nigdy.
  \item Chociaż nigdy jest często lepsze niż natychmiast.
  \item Jeśli rozwiązanie jest trudno wyjaśnić, to~jest ono złym pomysłem.
  \item Jeśli rozwiązanie jest łatwo wyjaśnić, to~może ono być dobrym pomysłem.
  \item Przestrzenie nazw to~jeden z niesamowicie genialnych pomysłów -- miejmy ich więcej!
\end{enumerate}

Ze względu na~to, że~język Ruby rządzi się nieco innymi zasadami niż język Python, pokusiłem się o~napisanie \textit{The Zen of Ruby} \footnote{\url{https://github.com/placek/ruby\_tao/blob/master/zen\_of\_ruby.txt}}:

\begin{enumerate}
  \item Beautiful is better than ugly.
  \item Explicit is better than implicit.
  \item Simple is better than complex.
  \item Complex is better than complicated.
  \item Flat is better than nested.
  \item Sparse is better than dense.
  \item Readability counts.
  \item Special cases aren't special enough to break the rules.
  \item Errors should never pass silently.
  \item Unless explicitly silenced.
  \item In the face of ambiguity, refuse the temptation to guess.
  \item There's more than one way to do it.
  \item Although all of them may be inappropriate.
  \item Now is better than never.
  \item Although never is often better than right now.
  \item If the implementation is hard to explain, it's a bad idea.
  \item If the implementation is easy to explain, it may be a good idea.
  \item Divide and abbreviate the code.
\end{enumerate}

Oraz w~wersji po~polsku:

\begin{enumerate}
  \item Piękne jest lepsze niż brzydkie.
  \item Wyrażone wprost jest lepsze niż domniemane.
  \item Proste jest lepsze niż złożone.
  \item Złożone jest lepsze niż skomplikowane.
  \item Płaskie jest lepsze niż wielopoziomowe.
  \item Rzadkie jest lepsze niż gęste.
  \item Czytelność się liczy.
  \item Sytuacje wyjątkowe nie są~na~tyle wyjątkowe, aby łamać reguły.
  \item Błędy zawsze powinny być sygnalizowane.
  \item Chyba że zostaną celowo ukryte.
  \item W razie niejasności powstrzymaj pokusę zgadywania.
  \item Istnieje wiele sposobów na~rozwiązanie danego problemu.
  \item Aczkolwiek każdy może być nie najlepszym.
  \item Teraz jest lepsze niż nigdy.
  \item Chociaż nigdy jest często lepsze niż natychmiast.
  \item Jeśli rozwiązanie jest trudno wyjaśnić, to~jest ono złym pomysłem.
  \item Jeśli rozwiązanie jest łatwo wyjaśnić, to~może ono być dobrym pomysłem.
  \item Dziel i~zdobywaj!
\end{enumerate}

\subsection{Komentarze} \label{dokumentacja.komentarze}

Komentarze w~kodzie to~dobrzy sposób na~wyjaśnienie zasadności użycia algorytmów, bądź struktury API poszczególnych elementów kodu. Stosowanie komentarzy pozwala także wyróżnić co~ważniejsze dla ,,czytelnika'' elementy kodu.

\subsubsection{RDoc} \label{dokumentacja.rdoc}

Istnieje wiele konwencji co~do~formy komentarzy. Format komentarzy ma~znaczenie nie tylko podczas czytania kodu -- może on~posłużyć pośrednio do~wygenerowania pełnej dokumentacji API przy użyciu odpowiednich narzędzi.


Komentarze w~języku Ruby mogą być dwojakiego formatu:

  \lstset{language=Ruby, caption=Komentarze w~języku Ruby., basicstyle=\ttfamily\footnotesize, numbers=left, numberstyle=\footnotesize, captionpos=b, backgroundcolor=\color{LightGray}}
  \begin{lstlisting}
    # Komentarz jednolinijkowy rozpoczyna sie
    # od znaku krzyzyka.

    =begin
      Blok komentarza.
      Taki blok rozpoczyna sie od znacznika "=begin"
      a konczy na znaczniku "=end".
      Komentarze jednolinijkowe sa jednak czesciej
      uzywane.
    =end
  \end{lstlisting}


Narzędzie \textit{RDoc}\footnote{Patrz: \url{rdoc.sourceforge.net}} to~narzędzie konsolowe generujące dokumentację API w~zadanym formacie (standardowo jest to~HTML) na~podstawie kodu oraz~komentarzy w~nim zawartych. Aby wygenerować taką dokumentację wystarczy wpisać w~konsoli:

\mbox{\texttt{\$ rdoc <opcje> [plik...] }}

\textit{RDoc} Wygeneruje dla nas czytelną dokumentację nawet jeśli nie opatrzymy kodu komentarzami. Mimo to~lepiej jest napisać takie komentarze. \textit{RDoc} oferuje specyficzny format treści komentarzy:

\begin{enumerate}
  \item Listy (wypunktowanie) tworzymy poprzez ustawienie na~początku linii gwiazdki~\texttt{*} bądź minusa \texttt{-}.\\
    \texttt{- Pierwszy punkt.\\- Drugi punkt.\\- \ldots}
  \item Listy numerowane tworzymy poprzez ustawienie numeru oraz~kropki na początku linii.\\
    \texttt{1. Pierwszy punkt.\\2. Drugi punkt.\\3. \ldots}
  \item Słowniki tworzymy poprzez wstawienie dwóch dwukropków po~tłumaczonym słowie.\\
    \texttt{kot:: Małe domowe zwierzątko.\\tygrys::    Nieco większe dzikie zwierzątko.}
  \item Nagłówki generowane są~poprzez wstawienie znaku \texttt{=} na początku linii. Mogą istnieć trzy poziomy nagłówków.\\
    \texttt{= Nagłówek pierwszego rzędu\\== Nagłówek drugiego rzędu\\=== Nagłowek trzeciego rzędu}
  \item Poziomą linię wstawiamy poprzez podanie sekwencji co~najmniej trzech minusów \texttt{-}.\\
    \texttt{-----------}
  \item Kursywę, pogrubienie i~podkreślenie możemy zastosować na~słowie otaczając słowo podkreślnikiem \texttt{\_}, gwiazdką \texttt{*}, bądź plusem \texttt{+}.\\
    \texttt{\_italic\_  *bold*  +underline+}
\end{enumerate}

Ponadto możemy użyć dodatkowych znaczników by~kontrolować wygląd naszej dokumentacji w~docelowym formacie (np.~w~HTML).

\begin{enumerate}
  \item Nazwy klas, metod występujące w~komentarzach zamieniane są~automatycznie na~odnośniki do~konkretnych miejsc w~dokumentacji.
  \item Odnośniki zaczynające się od: \texttt{http:}, \texttt{mailto:} albo \texttt{www.} są~rozpoznawane jako linki. Odnośnik HTTP, który wskazuje na obrazek wstawia obrazek w linijce. Adresy zaczynające się od \texttt{link:} traktowane są~jako odnośniki do~lokalnych plików.
  \item Hiperłącza mogą być także tworzone według schematu:\\
    \textit{nazwa}\texttt{[}\textit{url}\texttt{]}
  \item Słowo kluczowe \texttt{:title:} pozwala ustawić tytuł strony dokumentacji.
\end{enumerate}

\subsection{Testy} \label{dokumentacja.testy}

Testy to~dodatkowe aplikacje sprawdzające poprawność logiki naszego projektu. Zasadniczą ich funkcją jest dowiedzenie, że~dana funkcjonalność została zaimplementowana.


Testować w~naszej aplikacji możemy wszystko, jednak dobrze jest ustalić co~chcemy uzyskać poprzez napisanie testów. Biorąc pod uwagę powyższe kryterium testy dzielimy na:

\begin{enumerate}
 \item testy jednostkowe. Testami jednostkowymi nazywamy testy, które sprawdzają poprawność modułów ,,silnika'' aplikacji. Za~pomocą testów jednostkowych testuje się zwykle klasy, metody, stany maszyny, poprawność algorytmów, wejścia i~wyjścia strumieni itp.
 \item testy behawioralne. Testy takie odpowiadają na~pytanie: ,,co~się stanie, gdy zrobimy \ldots''. Testują one zachowanie aplikacji -- reakcję na~żądania, efekty działania zdarzeń (takich jak wypełnienie formularza) itp.
\end{enumerate}

\subsubsection{Testy jednostkowe RSpec} \label{dokumentacja.rspec}

\textit{RSpec}\footnote{Patrz: \url{rspec.info}} jest narzędziem do~tworzenia testów jednostkowych dla aplikacji napisanych w~języku Ruby. Jego składnia jest prosta, a~testy napisane w~nim -- czytelne:

  \lstset{language=Ruby, caption=Test RSpec testujący klasę \texttt{Array}., basicstyle=\ttfamily\footnotesize, numbers=left, numberstyle=\footnotesize, captionpos=b, backgroundcolor=\color{LightGray}} \label{code.prostytest}
  \begin{lstlisting}
  describe Array do
    describe "#push" do
      it "puts a value at the end of array" do
        array = Array.new
        value = "Some value"
        array.push value
        array.last.should == value
      end
    end
  end
  \end{lstlisting}

W~pierwszej kolejności podajemy opis testu -- krótką informację o~tym co~testujemy i~jakie mamy oczekiwania względem testowanego obiektu. Wewnątrz bloku testu realizujemy przypadek użycia naszego obiektu. W~każdym momencie możemy sprawdzić, czy interesujący nas stan obiektu jest zgodny z~naszymi oczekiwaniami. W~tym celu używamy metod \texttt{should} oraz~\texttt{should\_not}.


Testy uruchamiamy podając w~konsoli polecenie \texttt{rspec} oraz~plik z~napisanymi przez nas testami:

\mbox{\texttt{\$ rspec nazwa\_pliku\_testu.rb}}

Trzeba pamiętać, że~testy powinny mieć dostęp do~logiki naszego projektu -- np.~poprzez użycie instrukcji \texttt{require}. Efektem działania naszego testu (z~przykładu \ref{code.prostytest}) będzie:

  \lstset{language=bash, caption=Efekt uruchomienia testu jednostkowego z~poprzedniego przykładu., basicstyle=\ttfamily\footnotesize, numbers=left, numberstyle=\footnotesize, captionpos=b, backgroundcolor=\color{LightGray}}
  \begin{lstlisting}
.

Finished in 0.00258 seconds
1 example, 0 failures
  \end{lstlisting}

Aby zrozumieć, co~stało się podczas testowania, możemy zmienić format wyjścia na~bardziej przyjazny człowiekowi (użyjemy do~tego opcji \texttt{----format d}):

  \lstset{language=bash, caption=Efekt uruchomienia testu jednostkowego z~opcją \texttt{----format d}., basicstyle=\ttfamily\footnotesize, numbers=left, numberstyle=\footnotesize, captionpos=b, backgroundcolor=\color{LightGray}}
  \begin{lstlisting}
Array
  #push
    puts a value at the end of array

Finished in 0.0029 seconds
1 example, 0 failures
\end{lstlisting}

Jak widać \texttt{rspec} poinformował nas o~tym, które testy ,,przeszły'' -- czyli, które z~testowanych przez nas funkcjonalności spełniły nasze oczekiwania. Dla jasności podam jeszcze jeden przykład testów jednostkowych:

  \lstset{language=bash, caption=Efekt uruchomienia testu jednostkowego z~opcją \texttt{----format d}., basicstyle=\ttfamily\footnotesize, numbers=left, numberstyle=\footnotesize, captionpos=b, backgroundcolor=\color{LightGray}} \label{code.testzly}
  \begin{lstlisting}
describe Array do
  describe "#push" do
    it "puts a value at the end of an array" do
      array = Array.new
      value = "Some value"
      array.push value
      array.last.should == value
    end
    it "increases a size of an array of one" do
      array = Array.new
      array_size = array.size
      array.push "Some value"
      array.size.should == (array_size + 1)
    end
  end
  describe "#pop" do
    it "gets a value from the end of an array" do
      array = Array.new
      value = "Some value"
      array.push value
      array.pop.should == value
    end
    it "increases a size of an array of one" do
      array = Array.new
      array.push "Some value"
      array_size = array.size
      array.pop
      array.size.should == (array_size + 1)
    end
  end
end
  \end{lstlisting}

Uruchamiając testy poprzez podanie \texttt{rspec} bez opcji \texttt{----format d} otrzymujemy:

  \lstset{language=bash, caption=Efekt uruchomienia testu jednostkowego dla przykładu \ref{code.testzly}., basicstyle=\ttfamily\footnotesize, numbers=left, numberstyle=\footnotesize, captionpos=b, backgroundcolor=\color{LightGray}}
  \begin{lstlisting}

...F

Failures:
  1) Array#pop increases a size of an array of one
     Failure/Error: array.size.should == (array_size + 1)
     expected: 2,
          got: 0 (using ==)
     # ./array_spec.rb:28

Finished in 0.00951 seconds
4 examples, 1 failure
  \end{lstlisting}

Jak widać pewne testy nie przechodzą. Oznacza to, że~albo funkcjonalność nas interesująca nie jest zgodna z~naszym założeniem, albo oczekujemy od~niej zbyt wiele. Po uruchomieniu testów z opcją \texttt{----format d} otrzymujemy w~wyniku:

  \lstset{language=bash, caption=Efekt uruchomienia testu jednostkowego dla przykładu \ref{code.testzly} z~opcją \texttt{----format d}., basicstyle=\ttfamily\footnotesize, numbers=left, numberstyle=\footnotesize, captionpos=b, backgroundcolor=\color{LightGray}}
  \begin{lstlisting}
Array
  #push
    puts a value at the end of an array
    increases a size of an array of one
  #pop
    gets a value from the end of an array
    increases a size of an array of one (FAILED - 1)

Failures:
  1) Array#pop increases a size of an array of one
     Failure/Error: array.size.should == (array_size + 1)
     expected: 2,
          got: 0 (using ==)
     # ./array_spec.rb:28

Finished in 0.00692 seconds
4 examples, 1 failure
  \end{lstlisting}

\subsubsection{RSpec jako dokumentacja}


Przykład \ref{code.prostytest} mówi nam dość dużo o~sposobie użycia elementów logiki naszego projektu. Testy jednostkowe są~tak naprawdę (jak już wspomniałem) przypadkami użycia tychże elementów. Co~więcej -- przypadki te~realizują się zgodnie z~założeniami twórcy takiego projektu (uruchomione testy ,,przechodzą''). Oznacza to, że~są one dobrą dokumentacją działania i~sposobu użycia poszczególnych obiektów logiki projektu.

\subsubsection{Dokumentacja poprzez testy behawioralne} \label{dokumentacja.cucumber}

Skoro testy jednostkowe są~swego rodzaju dokumentacją, to~także testy behawioralne mogą nią być. Oczywiście ze~względu na~inny cel testy behawioralne będą dokumentować projekt z~innej perspektywy. Testy te~określają zachowanie aplikacji, a~zatem kwestii ściśle związanej z~użytkowaniem. W~praktyce testy behawioralne są~stosowane dla określenia wymagań ze~strony klienta -- zleceniodawcy projektu.


Istnieje kilka narzędzi, przy pomocy których można napisać i~przeprowadzić testy behawioralne. Można do~tego użyć \texttt{rspec} oraz~\texttt{Test::Unit} (czyli narzędzi wykorzystywanych przy pisaniu testów jednostkowych). Dla dobrej czytelności oraz~eleganckiego formatu używam narzędzia \textit{Cucumber}\footnote{Patrz: \url{cukes.info}}.

\subsection{System kontroli wersji} \label{dokumentacja.git}

Systemy kontroli wersji pozwalają na~współdzielenie kodu projektu oraz~jednoczesną pracę nad nim wielu programistom. Systemy takie przechowują także pełną historię zmian zachodzących w~projekcie, w~tym: informacje o~autorze zmian, informacje o~czasie modyfikacji, wersje plików przed i~po~modyfikacji itp. Informacje te~są~niezwykle cenne jako dokumentacja opisująca pracę zespołu. Na~podstawie logów można określić np.~trudność realizacji danego zagadnienia (na~podstawie czasu wykonania bądź dodanych linii kodu), aktywność programistów oraz~temu podobne.


Przykładową dokumentacją przebiegu pracy nad projektem może być log narzędzia \textit{Git}\footnote{Patrz: \url{git-scm.com}} wygenerowany za~pomocą polecenia:

\mbox{\texttt{\$ git log}}

  \lstset{language=bash, caption=Log narzędzia \textit{Git} dla pewnego projektu., basicstyle=\ttfamily\footnotesize, numbers=left, numberstyle=\footnotesize, captionpos=b, backgroundcolor=\color{LightGray}}
  \begin{lstlisting}

commit aaf288534af2ed15da3ca1b200025b8be4c3526b
Author: Pawel Placzynski <placek@example.com>
Date:   Fri Dec 10 14:47:45 2010 +0100

    Updated mongodb

commit 06fdd81b0104186f9f449487a9a1ceb92d4101f6
Author: Kamil Zygmuntowicz <kamil.zygmuntowicz@example.com>
Date:   Fri Dec 3 11:10:25 2010 +0100

    Added rspec on [#6412487]

commit d081a370bf60f8ef0b8678602b4745b0b5e00d39
Author: andst4@example.com <andrzej@andrzej-laptop.(none)>
Date:   Fri Dec 3 09:54:04 2010 +0100

    Copying query with name.

commit 65d7d3daa13eabba81ed23cad8117c59e6a44e74
Author: Sebastian Wojtczak <wojtczaksebastian@example.com>
Date:   Tue Nov 30 17:11:39 2010 +0100

    [6672367] Removed unnecessary border on app dashboard.

commit 818a69e07cf665021a4bb01cb5ed533020c8ada5
Author: Sebastian Wojtczak <wojtczaksebastian@example.com>
Date:   Sat Nov 27 21:32:03 2010 +0100

    [#6760943] Sort helper

  \end{lstlisting}

\subsection{UML} \label{dokumentacja.uml}

Ostatnią, chyba najlepszą dla tak zwanych ,,wzrokowców'', formą dokumentacji jest opisywanie struktur oraz~zachowań projektu przy pomocy diagramów UML. UML (z~ang.~Unified Modeling Language, czyli Zunifikowany Język Modelowania) jest formalnym językiem pozwalającym na modelowanie różnego rodzaju systemów (w~tym systemów informatycznych). Język ten jest używany w~większości przypadków raczej jako model przyszłego (mającego powstać) systemu a~rzadziej jako dokumentacja istniejącego już systemu. Jednak w~myśl zasad inżynierii wstecznej\footnote{Patrz: \url{http://www.npd-solutions.com/reoverview.html}} każda forma opisywania systemu jest jak najbardziej poprawną formą dokumentacji tegoż systemu.


Okazuje się, że~UML jest jedną z~bardziej czytelnych form dokumentacji, a~co~za~tym idzie -- szybką i~praktyczną w~użytkowaniu. Diagramy mają to~do~siebie, że~szybko wpadają w~pamięć, łatwiej przyswoić noszoną przez nie treść a~także zrozumieć co~bardziej skomplikowane aspekty tej treści.
