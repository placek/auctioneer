\section{Definicje} \label{definicje}

\subsection{Pojęcia związane z~inżynierią oprogramowania} \label{definicje.inz_op}

\begin{definition}[Aplikacja szkieletowa] \label{definicje.inz_op.framework}
W~programowaniu komputerowym aplikacja szkieletowa (framework) jest szkieletem do~budowy aplikacji. Definiuje on~strukturę aplikacji oraz~ogólny mechanizm jej działania, a~także dostarcza zestaw komponentów i~bibliotek ogólnego przeznaczenia do~wykonywania określonych zadań. Programista tworzy aplikację, rozbudowując i~dostosowując poszczególne komponenty do~wymagań realizowanego projektu, tworząc w~ten sposób gotową aplikację\cite{framework}.


Frameworki bywają niekiedy błędnie zaliczane do~bibliotek programistycznych. Typowe cechy, które każą wyróżniać je~jako samodzielną kategorię oprogramowania, to:

\begin{enumerate}
  \item Odwrócenie sterowania -- w~odróżnieniu od~aplikacji oraz~bibliotek, przepływ sterowania jest narzucany przez framework, a~nie przez użytkownika.
  \item Domyślne zachowanie -- framework posiada domyślną konfigurację, która musi być użyteczna i~dawać sensowny wynik, zamiast być zbiorem pustych operacji do~nadpisania przez programistę.
  \item Rozszerzalność -- poszczególne komponenty frameworka powinny być rozszerzalne przez programistę, jeśli ten chce rozbudować je~o~niezbędną mu~dodatkową funkcjonalność.
  \item Zamknięta struktura wewnętrzna -- programista może rozbudowywać framework, ale nie poprzez modyfikację domyślnego kodu.
\end{enumerate}
\end{definition}

\begin{definition}[Interpretowany język programowania] \label{definicje.inz_op.jez_inter}
Języki interpretowane to~języki programowania, które zazwyczaj implementowane są~w~formie interpretera, a~nie kompilatora. Program w~języku interpretowanym nie jest kompilowany, lecz jest przechowywany w postaci kodu źródłowego i~dopiero podczas uruchomienia wczytywany, interpretowany i~wykonywany przez interpreter języka.
\end{definition}

\begin{definition}[Paradygmat obiektowości] \label{definicje.inz_op.jez_obj}
Programowanie obiektowe (ang.~object-oriented programming) -- paradygmat programowania, w~którym programy definiuje się za~pomocą obiektów -- elementów łączących stan (czyli dane, nazywane najczęściej polami) i~zachowanie (czyli procedury, tu:~metody). Obiektowy program komputerowy wyrażony jest jako zbiór takich obiektów, komunikujących się pomiędzy sobą w~celu wykonywania zadań.


Podejście to~różni się od~tradycyjnego programowania proceduralnego, gdzie dane i~procedury nie są~ze~sobą bezpośrednio związane. Programowanie obiektowe ma ułatwić pisanie, konserwację i~wielokrotne użycie programów lub ich fragmentów.


Największym atutem programowania, projektowania oraz~analizy obiektowej jest zgodność takiego podejścia z~rzeczywistością -- mózg ludzki jest w~naturalny sposób najlepiej przystosowany do~takiego podejścia przy przetwarzaniu informacji.\cite{lang}
\end{definition}

\begin{definition}[Paradygmat funkcyjności] \label{definicje.inz_op.jez_fun}
Programowanie funkcyjne (lub programowanie funkcjonalne) to~filozofia i~metodyka programowania będąca odmianą programowania deklaratywnego, w~której funkcje należą do~wartości podstawowych, a~nacisk kładzie się na~wartościowanie (często rekurencyjnych) funkcji, a~nie na~wykonywanie poleceń.


W czystym programowaniu funkcyjnym, raz zdefiniowana funkcja zwraca zawsze tę~samą wartość dla danych wartości argumentów, tak jak prawdziwe funkcje matematyczne.


Inspiracją dla powstania programowania funkcyjnego był w~matematyce rachunek lambda.\cite{lang}
\end{definition}

\begin{definition}[Metaprogramowanie] \label{definicje.inz_op.metaprog}
Metaprogramowanie to~technika umożliwiająca programom tworzenie lub modyfikację kodu innych programów (lub ich samych). Program będący w~stanie modyfikować lub generować kod innego programu nazywa się metaprogramem.


Wykorzystanie zasad metaprogramowania pozwala na~przykład na~dynamiczną modyfikację programu podczas jego kompilacji.


Metaprogramy pisze się w~metajęzykach. Jeśli język jest jednocześnie swoim metajęzykiem, to~taką cechę nazywamy refleksyjnością (ang. reflexivity).\cite{lang}
\end{definition}

\begin{definition}[Metodyka] \label{definicje.inz_op.metodyka}
Metodyka to~ustandaryzowane dla wybranego obszaru podejście do~rozwiązywania problemów. Metodyka abstrahuje od~merytorycznego kontekstu danego obszaru, a~skupia się na~metodach realizacji zadań, szczególnie metodach zarządzania.


W~odróżnieniu od~metodologii, która się skupia na~odpowiedzi na~pytanie; ,,Co należy robić?'' metodyka koncentruje się na~poszukiwaniu odpowiedzi na~pytanie: ,,Jak to~należy robić?''. Generalnie metodyka bardziej kieruje się ku~praktyce wykonawczej, a~metodologia ku~teorii zazwyczaj sprawnego działania.
\end{definition}

\subsection{Pozostałe pojęcia}

\begin{definition}[Aukcja] \label{definicje.inne.aukcja}
Aukcja -- zorganizowana forma sprzedaży, będąca formą przetargu prowadzonego na~żywo. Zazwyczaj aukcje przeprowadza się wtedy, gdy~istnieje wielu potencjalnych nabywców na~jeden towar.


Aukcja internetowa -- rodzaj aukcji przeprowadzanej za~pośrednictwem Internetu.
\end{definition}
